\chapter{Plano de Atividades}

\lecture{1}{23 Março 2023}{Mensagem 1}

\section{Introdução}

O objetivo deste trabalho é aproximar a implementação de um ambiente com dispositivos Web of Things e conectar as ICs dos alunos do grupo. Para isso, foi avaliado um gateway para web of things chamado WoTPy. O artigo que o apresenta é "WOTPY: A framework for Web of Things applications", disponível para download na internet.

O WoTPy foi escolhido porque foi desenvolvido dentro de um grupo de pesquisa em Web Of Things, é codificado em Python e possui código aberto no Github. Além disso, tem potencial para ser usado em Semantic Web of Things (SWoT). No entanto, não está maduro o suficiente para ser simplesmente baixado e utilizado. Por isso, o objetivo do TCC é levar o WoTPy a um ponto de maturidade adequado.

Para alcançar esse objetivo, será feito um fork do WoTPy e ajustadas as dependências. Em seguida, será encapsulado em um container (Docker) e testado com um dispositivo baseado em ESP32. Se houver tempo suficiente, será implementada no WoTPy a versão mais recente da especificação da W3C para WoT.

Este trabalho é importante para o avanço da tecnologia Web of Things e para a conexão das ICs dos alunos do grupo. Com a conclusão do projeto, espera-se contribuir para a evolução do WoTPy e torná-lo uma ferramenta mais acessível para a comunidade.

\url{https://www.overleaf.com/read/bjszxjfvdpnk}
