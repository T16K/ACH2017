\lecture{6}{17 Maio 2023}{ESP32}

\section{Semana 06}

\subsection{Sensor Solar}

\url{https://t16k-ach2157.readthedocs.io/en/latest/}

Antes de começar, serão necessários os seguintes componentes:
\url{https://github.com/T16K/ACH2157#lista-de-materiais}
\begin{itemize}
    \item Sensor UV ML8511
    \item Microcontrolador ESP32
    \item Jumpers e protoboard para a conexão
    \item Ambiente de desenvolvimento Python configurado com WoTPy
\end{itemize}

Conectar o sensor UV ML8511 ao ESP32: 
\url{https://github.com/T16K/ACH2157#conex%C3%B5es}
\begin{itemize}
    \item GND (ML8511) para GND (ESP32)
    \item VCC (ML8511) para 3V3 (ESP32)
    \item OUT (ML8511) para um pino analógico 34 (ESP32)
\end{itemize}

Instalar a biblioteca do WoTPy.
\url{https://github.com/T16K/wot-py}

Escrever um código que leia os dados do sensor UV ML8511 e os exiba no ESP32, usando a linguagem de programação MicroPython, que é uma versão do Python 3 otimizada para microcontroladores como o ESP32.
\url{https://t16k-ach2157.readthedocs.io/en/latest/comp/esp.html#introduzindo-o-micropython}

Usar o WoTPy para expor os dados do sensor UV como um "Thing" na Web of Things. Isso permitirá que outros dispositivos e aplicações IoT interajam com seu sensor UV de maneira padronizada.

Testar as configuração para garantir que tudo esteja funcionando corretamente. É possível fazer isso usando o WoTPy para descobrir o sensor UV na rede e lendo os dados do sensor.

\subsection{Expor os Dados do Sensor}

Para expor os dados do sensor usando WoTPy, criar um Thing que represente o sensor. Cada Thing tem um conjunto de propriedades, ações e eventos que descrevem suas capacidades. Neste caso, criar uma propriedade para representar o valor lido do sensor UV.

Adicionar o WoTPy para expor esse valor como uma propriedade em um Thing. Exemplo simplificado:
 
\begin{lstlisting}[breaklines]
from wotpy.wot.servient import Servient
from wotpy.wot.thing import Thing
from wotpy.wot.forms import Form
from wotpy.protocols.http.server import HTTPServer

# Crie um novo Thing
uv_sensor_thing = Thing(id="urn:dev:ops:uv-sensor-1234", title="UV Sensor")

# Adicione uma propriedade ao Thing para representar o valor do sensor UV
uv_sensor_thing.add_property(name="uvValue", description={"@type": "number", "unit": "UV Index"}, value=0, forms=[Form(href="/uvValue", op="readproperty")])

# Este valor sera atualizado pelo seu codigo de leitura de sensor

# Crie um servient e adicione o Thing
servient = Servient()
servient.add_thing(uv_sensor_thing)

# Inicie o servidor HTTP para expor o Thing na rede
http_server = HTTPServer(port=8080)
http_server.start(servient)

# Agora o Thing esta disponivel em http://localhost:8080/uvValue
\end{lstlisting}

Este é um exemplo simplificado. Em um caso real, provavelmente teria que adicionar um loop que continuamente lê o valor do sensor UV e atualiza a propriedade uvValue.

Além disso, precisaria adicionar segurança e manipulação de erros apropriadas. Por exemplo, restringir o acesso à propriedade uvValue para apenas certos dispositivos ou usuários.

Finalmente, configurar o seu ESP32 para que ele envie os dados do sensor para o servidor WoTPy, possivelmente através de um protocolo como MQTT ou HTTP.

\subsection{cliente HTTP}

É necessário definir primeiro o tipo de entidade no nosso contexto: cliente ou servidor.

No caso de um servidor, a instalação do WoTPy é necessária. Note que para esta configuração, não há necessidade de um gateway.

Já para um cliente, um gateway se torna essencial. Este atua como intermediário entre o ESP32 e o mundo externo, facilitando a comunicação entre os dois.

Para comunicar os valores do sensor ML8511 usando o ESP32 e o WoTPy:

Leitura do sensor ML8511 com o ESP32: Primeiro, configurar o ESP32 para ler os valores do sensor ML8511. Programando o ESP32 com um código que lê os valores do sensor a intervalos regulares. Este código será executado no ambiente de desenvolvimento \textbf{MicroPython}.

Servidor no ESP32: O ESP32 é programado para funcionar como um servidor WoT. Ele expõe a propriedade que representa o valor do sensor ML8511 através de um protocolo de rede, como HTTP.

Comunicação de valores do sensor: Sempre que o ESP32 lê um valor do sensor, ele atualiza a propriedade do sensor no servidor WoT. Dependendo da sua implementação, isso pode envolver o envio de um POST request ao servidor WoTPy ou atualizar o valor localmente se o servidor WoT estiver sendo executado no próprio ESP32.

Criação de um Thing WoTPy para representar o sensor: No servidor WoTPy, criar um Thing que represente o sensor ML8511, utilizando a biblioteca WoTPy Python. O Thing deve ter uma propriedade que represente o valor atual do sensor.

Atualização do valor do sensor no Thing WoTPy: Cada vez que o ESP32 enviar um novo valor do sensor para o servidor WoTPy, atualizar o valor da propriedade no Thing WoTPy que representa o sensor. Isto pode ser feito manipulando a requisição POST que o ESP32 envia e atualizando o valor da propriedade apropriadamente.

Exposição do Thing WoTPy na rede: Finalmente, você configurar o WoTPy para expor o Thing na rede. Isto permitirá que outros dispositivos e serviços descubram o Thing e leiam o valor do sensor.

Interagindo com o Thing: Os clientes na rede, que também podem ser implementados usando o WoTPy ou qualquer outra biblioteca que suporte o padrão WoT, podem agora interagir com o Thing. Eles podem ler o valor atual do sensor, assinar atualizações de valor e, se suportado, acionar ações no Thing.
