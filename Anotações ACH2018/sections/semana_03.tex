\section{Semana 03}

\lecture{3}{21 Agosto 2023}{SSN}

Especificação do W3C para o "Semantic Sensor Network (SSN) Ontology", que é uma ontologia para descrever sensores e suas observações. A SSN é usada para representar diferentes aspectos de sistemas de sensoriamento, incluindo o hardware do sensor (como o próprio sensor e o sistema no qual ele está embutido), os processos de sensoriamento e os dados produzidos.

Os exemplos fornecidos no \url{https://www.w3.org/TR/vocab-ssn/#examples} mostram como usar a ontologia SSN para representar várias entidades e relacionamentos em um sistema de sensoriamento. Por exemplo, como representar um sensor, suas capacidades, a propriedade que ele observa, as condições sob as quais opera, etc.

\subsection{endpoint SPARQL}

Implementar um endpoint SPARQL no Tornado implica em criar um ponto de acesso em seu servidor Tornado que aceite consultas SPARQL e retorne resultados. Aqui está um exemplo básico de como você pode fazer isso, usando o Tornado junto com a rdflib para armazenar e consultar um grafo RDF.

Inicie o servidor e execute uma consulta SPARQL usando o parâmetro query na URL. Por exemplo: \url{http://localhost:8888/sparql?query=SELECT%20%3Fsubject%20%3Fobject%20WHERE%20%7B%3Fsubject%20%3Fpredicate%20%3Fobject%20.%7D}

\subsection{SSN}

Para implementar um endpoint SPARQL no Tornado que pode lidar com dados modelados usando a SSN, seria necessário uma abordagem semelhante à fornecida na resposta anterior, mas seu grafo RDF estaria preenchido com dados conforme a ontologia SSN.

Por exemplo, você pode ter um grafo RDF que contém informações sobre um termômetro, que é um sensor que mede a temperatura. Esse termômetro poderia ser modelado usando a ontologia SSN para indicar coisas como a propriedade que ele mede (temperatura), seu alcance de operação (por exemplo, -50°C a 150°C), a unidade de medida que usa (Celsius), etc.

Então, quando alguém envia uma consulta SPARQL para o seu endpoint, eles podem perguntar coisas como "Listar todos os sensores que medem temperatura" ou "Quais são os alcances operacionais de todos os termômetros no sistema?", e o seu servidor Tornado retornaria os resultados apropriados, com base nos dados em seu grafo RDF.

\subsection{Código}

\textbf{Ontologia SSN:}

Adicionei os namespaces para as ontologias ssn e sosa, que são partes essenciais da "Semantic Sensor Network Ontology". ssn é a ontologia original, enquanto sosa é uma ontologia simplificada e alinhada com ssn.

\textbf{Definições de Tipo e Classe:}

Defini UVSensor como uma instância de ssn.System. Na ontologia SSN, um sistema pode ser qualquer coisa, desde um simples sensor até uma plataforma inteira de sensores.

Criei uma classe UVObservation que é uma subclasse de Observation. Isso permite modelar cada leitura UV como uma observação específica.

Defini UVValue como um datatype. Isso permite que valores UV sejam armazenados e semânticamente associados como resultados de observações.

\textbf{Atualização do Grafo:}

Na função write\_uv, ajustei o método de atualização do grafo para usar conceitos da ontologia SSN.

Em vez de simplesmente armazenar uma leitura, agora modelamos uma "observação" realizada pelo sensor. Isso implica que o UVSensor "observa" uma UVObservation, que "tem um resultado" (HasResult) que é o valor UV.

A URI da observação é dinamicamente criada com base no valor UV, garantindo que cada observação seja única.

Removi as entradas antigas da observação antes de adicionar novas para evitar um acúmulo de observações antigas no grafo.

\textbf{Serialização:}

A serialização final do grafo em formato turtle permite que você veja como as informações são representadas usando a ontologia SSN.
Essas mudanças garantem que os dados do sensor UV sejam representados de uma forma que esteja em conformidade com a ontologia SSN, permitindo uma integração mais rica e significativa com outros sistemas que compreendem e utilizam essa ontologia.

\subsection{Reunião}

Implementando uma Observação no Formato RDF

Uma observação em RDF (Resource Description Framework) refere-se a uma maneira estruturada de representar informações. Vamos destrinchar uma observação e entender o que cada parte dela significa:

\begin{verbatim}
<Observation/346345> rdf:type sosa:Observation ;
\end{verbatim}

Aqui, estamos criando uma nova observação com o identificador <Observation/346345>. A relação rdf:type especifica que esta é uma instância da classe sosa:Observation.

\begin{verbatim}
sosa:observedProperty <sensor/35-207306-844818-0/BMP282/atmosphericPressure> ;
\end{verbatim}

Esta tripla indica qual propriedade foi observada. Neste caso, a pressão atmosférica é a propriedade observada pelo sensor BMP282.

\begin{verbatim}
sosa:hasFeatureOfInterest  <earthAtmosphere> ;
\end{verbatim}

Aqui, estamos especificando qual é o objeto ou entidade de interesse da observação, neste caso, a atmosfera terrestre.

\begin{verbatim}
sosa:madeBySensor <sensor/35-207306-844818-0/BMP282> ;
\end{verbatim}

Indica qual sensor fez a observação. Neste caso, é o sensor BMP282.

\begin{verbatim}
sosa:hasResult [
  rdf:type qudt-1-1:QuantityValue ;
  qudt-1-1:numericValue "101936"^^xsd:double ;
  qudt-1-1:unit qudt-unit-1-1:Pascal ] ;
\end{verbatim}

Aqui, estamos representando o resultado da observação. O valor observado é de 101936 Pascais.

\begin{verbatim}
sosa:resultTime [
  rdf:type time:Instant ;
  time:inXSDDateTimeStamp "2017-06-06T12:36:13+00:00"^^xsd:dateTimeStamp ] .
\end{verbatim}

Por último, estamos indicando o momento exato em que a observação foi realizada, codificado no padrão ISO8601.

Dado que as triplas estão armazenadas em um servidor SPARQL, você pode acessá-lo, por exemplo, através de uma interface Tornado, que retorna uma página SPARQL. Em seguida, pode-se escrever uma consulta em SPARQL para recuperar informações específicas. Ao executar essa consulta, o servidor retornará uma lista de resultados correspondentes à consulta.
