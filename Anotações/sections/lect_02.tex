\lecture{2}{27 Março 2023}{Reunião 1}

\section{Apresentação}

O objetivo deste trabalho é implementar uma recomendação da W3C em um ambiente acadêmico. Diferente dos softwares comerciais, o foco está na interface do usuário. Embora a aderência aos padrões seja importante, o objetivo principal é construir um ambiente que atenda às necessidades dos usuários.

Para atingir esse objetivo, será utilizado o WoTPy, um software desenvolvido na área acadêmica capaz de implementar um gateway web of things. O WoTPy está dentro da recomendação da W3C, o que garante a qualidade e segurança da implementação.

É importante destacar que a implementação da recomendação da W3C trará diversos benefícios para o ambiente acadêmico. Isso inclui a melhoria da qualidade e segurança dos dados, a facilidade de integração de novos dispositivos e a padronização das comunicações entre eles. Além disso, a implementação da recomendação da W3C permitirá que o ambiente acadêmico esteja em conformidade com os padrões internacionais, o que é importante para a reputação da instituição.

\href{https://www.w3.org/TR/wot-architecture11/#expose-consumed-thing}{Exposed Thing and Consumed Thing Abstractions}

\section{Problema}

O objetivo deste projeto é explorar o potencial do WoTPy, uma ferramenta para implementação de gateways em Web of Things, e justificar por que ela é interessante para a nossa aplicação. Durante o primeiro semestre, vamos trabalhar na adaptação de dispositivos para conversar com o WoTPy e testar sua funcionalidade internamente. Também vamos procurar soluções para os problemas encontrados, contribuindo assim para a comunidade que criou a ferramenta.

Para facilitar o uso do WoTPy, vamos implementá-lo em um contêiner Docker. Além disso, vamos explorar a possibilidade de gerar automaticamente o código necessário para adaptar dispositivos ao WoTPy, aumentando assim sua usabilidade.

Durante o segundo semestre, vamos aprofundar nosso conhecimento do WoTPy e desenvolver ferramentas para geração de código e criação de grafos de conhecimento. Essas ferramentas podem ajudar a simplificar ainda mais a implementação de gateways em Web of Things e aumentar sua aderência a padrões.

\begin{itemize}
    \item quarta-feira 17:30
    \item 07/04 ultima versão pronta 
    \item 31/03 versão intermediária
\end{itemize}

