\lecture{3}{29 Março 2023}{Reunião 2}

\section{Objetivo}

\url{https://www.w3.org/WoT/IG/wiki/Terminology}

O objetivo específico deste projeto é colocar o WoTPy para funcionar, suas extensões e casos específicos, durante o primeiro semestre, e posteriormente, no segundo semestre, integrá-lo com outras ferramentas, como a geração de código e a utilização de GPT (Generative Pre-trained Transformer). Para isso, será necessário compreender melhor o contexto mais amplo do WoT e suas definições.

Web of Things (WoT) é uma área que busca a interoperabilidade entre diferentes dispositivos, permitindo que eles se comuniquem utilizando diferentes protocolos e atendam a diferentes requisitos. Desde 2010, pesquisadores como Dominique Dagardi e Ciocia têm trabalhado em soluções para o desenvolvimento de dispositivos semânticos, ou seja, que possam interpretar dados de diferentes máquinas e realizar ações de forma autônoma.

Uma das iniciativas mais importantes na área do WoT é a recomendação da W3C (World Wide Web Consortium), que busca estabelecer uma terminologia comum para a área e identificar padrões para a interoperabilidade. Além disso, a cronologia do IoT mostra que a interoperabilidade é um problema não resolvido desde 2007, quando diferentes dispositivos começaram a ser utilizados para coletar e transmitir dados.

Para resolver este problema, uma das soluções propostas é a utilização de agentes de software, conhecidos como gateways, que possibilitam a comunicação entre diferentes dispositivos utilizando diferentes protocolos. O WoTPy é uma ferramenta voltada para a área acadêmica que permite a implementação de gateways WoT e sua extensão para diferentes casos específicos.

Durante o primeiro semestre, o objetivo será colocar o WoTPy para funcionar, utilizando o Docker para sua instalação e realizando testes internos para verificar seu desempenho. Também serão construídos dispositivos adaptados para conversar com o WoTPy, utilizando a geração automática de código.

No segundo semestre, serão exploradas outras ferramentas para a geração de código, como a utilização de grafos de conhecimento, e será realizada a integração com outras ferramentas, como o GPT e o OpenAPI, buscando aprimorar a capacidade do WoTPy de integrar diferentes dispositivos e protocolos.
