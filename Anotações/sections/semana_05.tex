\lecture{5}{10 Maio 2023}{Especificações e Dependências}

\section{Semana 05}

\subsection{Especificações do W3C-WoT}

\url{https://www.w3.org/TR/wot-architecture11/}

\url{https://www.w3.org/TR/wot-usecases/}

\url{https://www.w3.org/TR/wot-thing-description/}

WoT Thing Description [\href{https://www.w3.org/TR/wot-thing-description11/}{WOT-THING-DESCRIPTION}]: fornece um formato de dados legível por máquina para descrever a metadados e interfaces voltadas para a rede de Things.

WoT Binding Templates [\href{https://www.w3.org/TR/wot-binding-templates/}{WOT-BINDING-TEMPLATES}]: oferece diretrizes informativas sobre como definir interfaces voltadas para a rede em Things para protocolos específicos e ecossistemas IoT.

WoT Discovery [\href{https://www.w3.org/TR/wot-discovery/}{WOT-DISCOVERY}]: define um mecanismo de distribuição para metadados WoT (Thing Descriptions).

WoT Scripting API [\href{https://www.w3.org/TR/wot-scripting-api/}{WOT-SCRIPTING-API}]: habilita a implementação da lógica de aplicação de um Thing usando uma API JavaScript comum semelhante às APIs do navegador da Web.

WoT Security and Privacy Guidelines [\href{https://www.w3.org/TR/wot-security/}{WOT-SECURITY}]: fornece diretrizes para a implementação segura e configuração de Things e discute questões que devem ser consideradas em sistemas que implementam a W3C WoT.

\subsection{Dependências do W3C-WoTPy}

O arquivo setup.py define um conjunto de dependências de terceiros necessárias para a execução da biblioteca WotPy. Algumas dessas dependências são obrigatórias, enquanto outras são opcionais e só serão instaladas se estiverem disponíveis no sistema.

A seguir, estão as principais dependências definidas no arquivo setup.py:

tornado: uma biblioteca assíncrona usada para criar aplicativos Web em Python. É usada pela WotPy para criar servidores HTTP e WebSocket.

jsonschema: uma biblioteca que valida esquemas JSON e os dados JSON correspondentes. A WotPy usa essa biblioteca para validar as descrições de coisa (Thing Descriptions) recebidas e produzidas.

six: uma biblioteca que permite escrever código Python 2 e Python 3 compatível com ambos os ambientes. É usada pela WotPy para garantir a compatibilidade com as duas versões do Python.

rx: uma biblioteca de programação reativa que permite escrever código que responde a mudanças de estado de forma assíncrona. É usada pela WotPy para suportar a API WoT Scripting e as interações com as propriedades e eventos observáveis.

python-slugify: uma biblioteca que converte strings para "slug", um formato de texto que usa somente caracteres ASCII, números e traços. É usada pela WotPy para criar identificadores únicos para as coisas.

Existem também algumas dependências opcionais que a WotPy usa se estiverem instaladas no sistema:

aiocoap: uma biblioteca Python para o protocolo de transferência de dados Constrained Application Protocol (CoAP), usada pela WotPy para suportar o protocolo CoAP.

hbmqtt: uma biblioteca Python para implementar o protocolo Message Queue Telemetry Transport (MQTT), usada pela WotPy para suportar o protocolo MQTT.

websockets: uma biblioteca Python para suportar a comunicação WebSocket, usada pela WotPy para suportar o protocolo WebSocket.

zeroconf: uma biblioteca Python para suportar o protocolo DNS Service Discovery (DNS-SD), usada pela WotPy para descobrir serviços e dispositivos na rede.

Essas dependências são verificadas em tempo de execução e adicionadas ao conjunto de dependências, se estiverem disponíveis no sistema.
