\lecture{9}{12 Junho 2023}{}

\section{Semana 09}

O objetivo é estabelecer uma comunicação básica entre o ESP32 e o servidor, utilizando HTTP para transmitir os dados do sensor UV.

\begin{itemize}
    \item O script boot.py no ESP32 será responsável por estabelecer a conexão com a rede WiFi.
    \item O script main.py no ESP32 será encarregado de ler o valor do sensor UV e transmitir esse valor para o servidor por meio de um pedido HTTP POST.
    \item O servidor, que executa o script server.py, receberá e processará os valores do sensor UV que são enviados pelo ESP32.
    \item \url{https://github.com/T16K/wot-py/commit/34afc51099e2bec61f61d9fc7daf971a54dbc856}
\end{itemize}

\subsection{Servidor}

Encontrei problemas na abertura das portas do servidor. Para solucionar isso, testei o programa server.py a partir do exemplo \textit{temperature}, com o objetivo de acessar o servidor WoT (Web of Things) criado. O servidor escuta nas seguintes portas especificadas:

\begin{itemize}
\item Porta 9393 para conexões WebSocket.
\item Porta 9494 para conexões HTTP.
\end{itemize}

Contudo, ao tentar realizar uma solicitação GET para \url{http://localhost:9494/}, a conexão com o localhost foi recusada.

Essa situação não havia ocorrido antes de começar a executar os scripts dentro da imagem do Docker, então deduzi que este poderia ser o problema. Após pesquisar, encontrei a solução no link \url{https://docs.docker.com/engine/reference/run/#expose-incoming-ports}.

Portanto, modifiquei o comando original para: 
\begin{lstlisting}[breaklines]
docker container run -p 9090:9090 -p 9393:9393 -p 9494:9494 -ti --rm -v $PWD:/asdf 7af5d5d07d93 sh
\end{lstlisting}


No entanto, percebi outro problema: a configuração de rede do Docker não estava utilizando a mesma rede local do ESP32. Para diagnosticar esse problema, utilizei o comando "ip a". Para resolver, bastou utilizar a opção "--network host" e excluir a opção "-p".

\subsection{Função POST}

Com relação à função POST, estou enfrentando dois problemas principais. No terminal do servidor, estou recebendo as seguintes mensagens:

\begin{lstlisting}[breaklines]
404 GET /uv (::1) 0.64ms
404 GET /favicon.ico (::1) 0.64ms
\end{lstlisting}

E, no rshell, ao usar o comando "repl" para comunicar com o ESP32, obtenho:

\begin{lstlisting}[breaklines]
[Errno 113] ECONNABORTED
[Errno 118] EHOSTUNREACH
\end{lstlisting}
