\section{Contexto do trabalho}

\begin{frame}{Contexto do trabalho}
    
    % Área: Web das Coisas (Web of Things), Interoperabilidade, Internet das Coisas (IoT), Computação Física.
    \begin{block}{Área}
        Web das Coisas (WoT)
    \end{block}

    % Problema de Pesquisa: A busca por interoperabilidade entre dispositivos em redes IoT é um desafio persistente. As soluções atuais são fragmentadas, e existe um risco de não interoperabilidade devido a soluções proprietárias. A arquitetura proposta pelo W3C para a Web das Coisas é um passo em direção à resolução deste problema.
    \begin{block}{Problema/Questão de Pesquisa}
        Abordagem da interoperabilidade na Web das Coisas seguindo as recomendações do World Wide Web Consortium (W3C), com um foco particular na implementação em gateways.
    \end{block}

    % Motivação: A popularidade crescente da IoT e a necessidade de padronização e interoperabilidade entre dispositivos. O trabalho se inspira na iniciativa do W3C de criar padrões abertos para a Web das Coisas e busca contribuir para essa área de pesquisa. 
    \begin{block}{Motivação}
        Desafios técnicos e econômicos relacionados à interoperabilidade entre redes de Internet of Things (IoT), a necessidade de aderir a padrões reconhecidos para evitar a fragmentação e a importância de gateways no contexto de WoT.
    \end{block}

%    \begin{block}{Ponto de partida}
%        WoTPy é uma implementação experimental, na linguagem Python, de um \textit{runtime} (\url{https://www.w3.org/TR/wot-architecture11/\#wot-runtime}). Um \textit{runtime} é um programa que implementa tanto a interface Cliente quanto a interface Servidor, com as interfaces padronizadas pela recomendação. WoTPy, atualmente, está sem manutenção e, talvez, desatualizado (\url{https://github.com/agmangas/wot-py}). Pretende-se recuperar WoTPy e usá-lo ou personalizá-lo para uma aplicação conhecida. No caso, o dispositivo "Protetor Solar"
%    \end{block}

\end{frame}