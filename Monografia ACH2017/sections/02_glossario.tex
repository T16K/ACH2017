% ----------------------------------------------------------
% Glossário
% ----------------------------------------------------------
%
% Consulte o manual da classe abntex2 para orientações sobre o glossário.
%
%\glossary



% Aqui as palavras aparecerão em ordem alfabética. A palavra ou sigla a ser definida aparecerá em negrito seguida de dois pontos (:), e em seguida a definição é escrita sem negrito. Ex:
%
% GLCE: Gramática Livre de Contexto Estocástica – gramática livre de contexto com uma distribuição de probabilidades sobre as produções com o mesmo lado esquerdo.
%
% No caso de abreviaturas (siglas), mesmo descrevendo-as aqui não deixe de defini-las na primeira vez em que elas são empregadas!

\newglossaryentry{WoTPy}{
name={WoTPy:},
description={Web das Coisas Python (\textit{Web of Things Python}) - \textit{gateway} experimental baseado no W3C-WoT}
}

\newglossaryentry{gateway}{
name={Gateway:},
description={dispositivo ou \textit{software} que conecta redes ou sistemas diferentes, permitindo a comunicação e a transferência de dados entre eles}
}

\newglossaryentry{IoT}{
name={IoT:},
description={Internet das Coisas (\textit{Internet of Things}) - interconexão de dispositivos físicos (coisas) que são capazes de coletar e trocar dados por meio de uma rede}
}

\newglossaryentry{protocolos de comunicação}{
name={Protocolo de comunicação:},
description={conjuntos de regras e formatos de dados que governam a troca de informações entre sistemas ou dispositivos de rede}
}

\newglossaryentry{grafos de conhecimento}{
name={Grafo de conhecimento:},
description={estruturas de dados que representam conhecimento e informações em forma de grafos, onde os nós representam entidades e as arestas representam as relações entre elas}
}

\newglossaryentry{endpoint SPARQL}{
name={Endpoint SPARQL:},
description={interface de acesso a um grafo de conhecimento que permite consultas e recuperação de informações usando a linguagem de consulta SPARQL}
}

\newglossaryentry{interoperabilidade}{
name={Interoperabilidade:},
description={capacidade de diferentes sistemas ou dispositivos se comunicarem, trocarem informações e trabalharem juntos de forma eficiente e eficaz}
}

\newglossaryentry{interconexão}{
name={Interconexão:},
description={conexão e integração de diferentes sistemas, dispositivos ou redes para permitir a comunicação e a troca de informações entre eles}
}

\newglossaryentry{W3C}{
name={W3C:},
description={\textit{World Wide Web Consortium} - consórcio internacional que desenvolve padrões e diretrizes para a \textit{World Wide Web}, visando promover sua acessibilidade, usabilidade e interoperabilidade}
}

\newglossaryentry{WoT}{
name={WoT:},
description={Web das Coisas (\textit{Web of Things}) - extensão da Web para abranger a interconexão e interação de dispositivos físicos por meio de padrões e tecnologias da Web}
}

\newglossaryentry{Descrição de Coisas}{
name={Descrição de Coisas:},
description={\textit{Thing Descriptions} - descrições de metadados e interfaces de rede de coisas (dispositivos) no contexto do W3C-WoT}
}

\newglossaryentry{Modelos de Vinculação}{
name={Modelos de Vinculação:},
description={\textit{Binding Templates} - modelos que fornecem orientações para definir interfaces de rede para protocolos específicos e ecossistemas de IoT no contexto do W3C-WoT}
}

\newglossaryentry{API}{
name={API:},
description={\textit{Application Programming Interface} - conjunto de regras e protocolos que permite que diferentes softwares se comuniquem entre si}
}

\newglossaryentry{Scripting API}{
name={Scripting API:},
description={API baseada em JavaScript que permite a implementação da lógica de aplicação das coisas no contexto do W3C-WoT}
}

\newglossaryentry{Descoberta WoT}{
name={Descoberta WoT:},
description={\textit{WoT Discovery} - mecanismo de descoberta de recursos e serviços na arquitetura do WoT}
}

\newglossaryentry{Diretrizes de Segurança e Privacidade}{
name={Diretrizes de Segurança e Privacidade:},
description={\textit{Security and Privacy Guidelines} - orientações e melhores práticas para a implementação segura de dispositivos e serviços no contexto do WoT}
}

\newglossaryentry{arquitetura}{
name={Arquitetura:},
description={estrutura geral e organização de um sistema ou aplicação, incluindo seus componentes, padrões e princípios subjacentes}
}

\newglossaryentry{Web}{
name={Web:},
description={\textit{World Wide Web} - sistema de informação e documentos interconectados, acessíveis por meio da Internet e navegadores da Web}
}

\newglossaryentry{microcontrolador}{
name={Microcontrolador:},
description={dispositivo eletrônico que incorpora um microprocessador, memória e periféricos em um único chip}
}

\newglossaryentry{servidor}{
name={Servidor:},
description={computador ou sistema que fornece serviços, recursos ou funcionalidades a outros dispositivos ou sistemas, conhecidos como clientes, por meio de uma rede}
}

\newglossaryentry{handlers}{
name={Handlers:},
description={componentes de \textit{software} responsáveis por lidar com solicitações, eventos ou tarefas específicas em um sistema ou aplicação}
}

\glsaddall
\printglossary

\cleardoublepage
