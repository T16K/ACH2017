\chapter{Introdução}

Neste capítulo inicial, a contextualização, o tema da pesquisa, a tese, os propósitos e a abordagem de investigação do projeto serão expostos. Além disso, uma descrição sucinta da disposição do documento será proporcionada.

\section{Contextualização}

Este projeto de estudo está vinculado à área de Internet das Coisas (IoT), correspondendo à conexão de equipamentos inteligentes em uma rede mundial. A IoT detém a capacidade de revolucionar vários ramos, como saúde, indústria, agricultura e cidades inteligentes, gerando vantagens como automação, economia de energia e monitoramento à distância.

Contudo, a comunicação entre dispositivos IoT de diferentes marcas ainda representa um obstáculo considerável. A ausência de padronização e a variedade de protocolos e interfaces impedem a interação e integração entre estes equipamentos. Isso restringe a habilidade de elaborar soluções extensas e escaláveis que aproveitem todo o potencial da IoT.

Diante deste panorama, o projeto de estudo propõe enfrentar o desafio da interoperabilidade, concentrando-se na execução e melhoria do WoTPy, um \textit{gateway} experimental baseado no W3C-WoT. A meta é desenvolver uma resposta que facilite a interação e a união entre dispositivos IoT de diferentes marcas, seguindo os padrões e orientações estipulados pelo W3C.

\section{Tema da Pesquisa}

A fundamentação para a execução deste projeto de pesquisa reside na exigência de vencer as barreiras da interoperabilidade na IoT. A ausência de padronização e a variedade de protocolos e interfaces impedem a interação e a integração entre os dispositivos IoT, limitando o potencial da tecnologia.

O questionamento da pesquisa reside em criar uma solução que promova a interoperabilidade entre dispositivos IoT de variadas marcas usando o WoTPy. A meta é permitir que estes dispositivos interajam e colaborem de forma transparente, possibilitando a construção de respostas abrangentes e escaláveis na IoT fazendo o uso do WoTPy.

\section{Tese}

Baseado na análise do tema da pesquisa, a suposição apresentada é que a execução e aprimoramento do WoTPy como um \textit{gateway} experimental fundado no W3C-WoT possa simplificar a interação e a união entre dispositivos IoT de variadas marcas. Crê-se que essa resposta, alinhada aos padrões e orientações do W3C, pode colaborar para a superação dos obstáculos da interoperabilidade na IoT.

\section{Propósitos}

O propósito central deste projeto de pesquisa é desenvolver e aperfeiçoar o WoTPy como um \textit{gateway} experimental fundado no W3C-WoT, visando aprimorar a interoperabilidade entre dispositivos IoT de variadas marcas. Para atingir este propósito central, os seguintes propósitos específicos foram estipulados:

\begin{itemize}
\item Entender as normas do W3C-WoT e suas principais partes, como Descrição das Coisas (\textit{Thing Descriptions}), Modelos de Vinculação (\textit{Binding Templates}) e \textit{Scripting API}.
\item Selecionar e investigar as bibliotecas e ferramentas necessárias para a execução do WoTPy.
\item Solucionar desafios de instalação e configuração do WoTPy, visando simplificar sua implantação em diferentes ambientes.
\item Realizar testes e validações para garantir a funcionalidade e qualidade do WoTPy.
\item Criar um exemplo de uso práticos do WoTPy e documentação para auxiliar desenvolvedores e usuários na implementação e utilização do \textit{gateway}.
\end{itemize}

\section{Abordagem de Pesquisa}

Este projeto de pesquisa adota um método de pesquisa aplicada, combinando elementos de pesquisa exploratória e desenvolvimento de \textit{software}. A metodologia incluiu as seguintes etapas:

\begin{enumerate}
\item Revisão bibliográfica sobre o W3C-WoT, IoT e interoperabilidade.
\item Análise das normas do W3C-WoT e estudo aprofundado de suas principais partes.
\item Seleção e estudo das bibliotecas e ferramentas relacionadas ao WoTPy.
\item Desenvolvimento e aperfeiçoamento do WoTPy, incluindo solução de desafios de instalação, testes e validações.
\item Elaboração de exemplos de uso práticos do WoTPy e criação de documentação detalhada.
\end{enumerate}

\section{Disposição do Documento}

Este documento está disposto da seguinte forma:

\begin{itemize}
\item Capítulo 1: Introdução - expõe a contextualização, a fundamentação, a suposição, os propósitos e a abordagem de pesquisa do projeto.
\item Capítulo 2: Objetivos - expõe os objetivos gerais e específicos do projeto.
\item Capítulo 3: Revisão Bibliográfica - aborda os conceitos fundamentais relacionados ao W3C-WoT, IoT e interoperabilidade.
\item Capítulo 4: Metodologia - descreve detalhadamente as etapas e os procedimentos adotados na execução do projeto.
\item Capítulo 5: Resultados - expõe os resultados obtidos durante a execução e aprimoramento do WoTPy.
\item Capítulo 6: Discussão - analisa e discute os resultados obtidos, destacando suas contribuições e limitações.
\item Capítulo 7: Conclusão - sintetiza os principais pontos discutidos no trabalho, expõe as conclusões alcançadas e sugere possíveis direções futuras.
\end{itemize}