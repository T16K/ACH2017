\chapter{Objetivos}

\section{Objetivo Geral}

O propósito principal deste projeto foi auxiliar no progresso do WoTPy \citeonline{gitwotpy:2022}, um \textit{gateway} experimental sustentado no W3C-WoT. Este \textit{gateway} visa incrementar a interoperabilidade entre equipamentos IoT de distintos fabricantes, promovendo a comunicação e a associação destes em uma única plataforma ou sistema.

\section{Objetivos Específicos}

Para alcançar o objetivo principal, os seguintes objetivos específicos foram definidos:

\begin{enumerate}
\item Avaliar as especificações do W3C-WoT \citeonline{Architecture} e entender os conceitos chaves e requisitos para a implementação do WoTPy;
\item Escolher e examinar as bibliotecas e instrumentos necessários para as dependências do WoTPy;
\item Solucionar dificuldades de instalação, assegurando que o WoTPy possa ser implementado e configurado sem problemas em diferentes contextos e sistemas;
\item Realizar testes e validações das melhorias realizadas;
\item Criar exemplos de uso e documentação minuciosa para auxiliar programadores e usuários na implementação do WoTPy em seus projetos de IoT;
\end{enumerate}

\section{Desdobramento}

Com a realização destes objetivos no primeiro semestre de 2023, espera-se tratar, no segundo semestre do mesmo ano, a integração do WoTPy com grafos de conhecimento. Este aprimoramento visa intensificar a interoperabilidade dos equipamentos IoT, permitindo a divulgação das habilidades dos sensores e suas observações por meio de um \textit{endpoint SPARQL}.


