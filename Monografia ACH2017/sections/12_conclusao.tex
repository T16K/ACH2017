\chapter{Conclusão}

Neste trabalho, foi abordada a solução de problemas e a criação de exemplos de uso da biblioteca WoTPy, um gateway experimental baseado no W3C-WoT. O objetivo principal foi contribuir para o desenvolvimento do WoTPy, visando melhorar a interoperabilidade e facilitar a comunicação e integração de dispositivos IoT de diferentes fabricantes.

Para atingir esse objetivo, foram realizadas diversas etapas ao longo do trabalho. Inicialmente, foram analisadas as especificações do W3C-WoT, compreendendo os principais conceitos e requisitos para a implementação do WoTPy. Em seguida, foram selecionadas e estudadas as bibliotecas e ferramentas necessárias para o desenvolvimento do projeto.

Durante o processo, foram identificados e resolvidos problemas de instalação do WoTPy, garantindo sua facilidade de implantação e configuração em diferentes ambientes e sistemas operacionais. Além disso, foram realizados testes e validações para garantir a qualidade e funcionalidade das contribuições feitas.

Um exemplo de uso prático do WoTPy foi desenvolvido, acompanhado por uma documentação detalhada, com o intuito de auxiliar desenvolvedores e usuários na implementação do WoTPy em seus projetos IoT. Essa documentação busca disseminar o conhecimento e facilitar a adoção do WoTPy pela comunidade.

Ao finalizar este trabalho, podemos afirmar que os resultados alcançados foram significativos. O WoTPy se tornou um gateway funcional e eficaz, capaz de melhorar a interoperabilidade entre dispositivos IoT de diferentes fabricantes. As contribuições realizadas, como a resolução de problemas de instalação, os testes e validações, e o desenvolvimento de exemplos de uso e documentação, agregaram valor ao projeto e facilitaram sua adoção e utilização.

Com base nos resultados obtidos e nas experiências adquiridas durante a realização deste trabalho, uma das áreas de desenvolvimento futuro é a integração do WoTPy com grafos de conhecimento. Essa integração visa aproveitar a estrutura semântica dos grafos para divulgar as capacidades dos sensores e suas observações por meio de um endpoint SPARQL. Dessa forma, seria possível realizar consultas semânticas para descobrir e explorar os recursos disponíveis nos dispositivos IoT, facilitando ainda mais a interoperabilidade dos dados coletados.

No geral, este trabalho contribuiu para avanços na área de Web das Coisas, promovendo a interoperabilidade e facilitando a integração de dispositivos IoT. Espera-se que o WoTPy e as contribuições realizadas possam ser adotados e utilizados por desenvolvedores e pesquisadores, impulsionando ainda mais o progresso nessa área em constante evolução.