% ---
% RESUMOS
% ---

% resumo em português
\setlength{\absparsep}{18pt} % ajusta o espaçamento dos parágrafos do resumo
\begin{resumo}

%-------------------------------------------------------------------------
% Comentário adicional do PPgSI - Informações sobre ``referência'':
% 
% Troque os seguintes campos pelos dados de sua Dissertação/Tese (mantendo a 
% formatação e pontuação):
%   - SOBRENOME
%   - Nome1
%   - Nome2
%   - Nome3
%   - Título do trabalho: subtítulo do trabalho
%   - AnoDeDefesa
%
% Mantenha todas as demais informações exatamente como estão.
% 
% [Não usar essas informações de ``referência'' para Qualificação]
%
% Para Tese de Doutorado: trocar "Dissertação (Mestrado em Ciências)" por "Tese (Doutorado em Ciências)".
%-------------------------------------------------------------------------
\begin{flushleft}
ARIGA, Gustavo Tsuyoshi. \textbf{Integração do WoTPy com Grafos de Conhecimento para Aprimorar a Interoperabilidade na Internet das Coisas (IoT)}. \imprimirdata. \pageref{LastPage} p. Monografia – Escola de Artes, Ciências e Humanidades, Universidade de São Paulo, São Paulo, 2023. \end{flushleft}

% Parágrafo único e no máximo 500 palavras. constituído de uma sequência de frases concisas e objetivas, em forma de texto. Deve apresentar os objetivos, métodos empregados, resultados e conclusões.

Neste trabalho, será abordada a integração do WoTPy com grafos de conhecimento, contribuindo para superar os desafios de interoperabilidade presentes na crescente área da Internet das Coisas (IoT). WoTPy, um arcabouço que pode ser usado para construir \textit{gateways IoT}, implementa parte do padrão \textit{W3C-Web of Things} (WoT) mas enfrenta desafios significativos em termos de integração e interpretação de dados provenientes de uma variedade de dispositivos IoT. O foco deste projeto é ampliar as funcionalidades do WoTPy por meio da integração com uma arquitetura de grafos de conhecimento, proporcionando  interpretação semântica dos dados da IoT. Por meio da aplicação de ontologias como SSN (Semantic Sensor Network) e SOSA (Sensor, Observation, Sample, and Actuator), o estudo aprofunda a representação e as interações entre os dispositivos IoT. A implementação de um modelo RDF/OWL, juntamente com um endpoint SPARQL, são etapas necessárias neste processo. Esta implementação tem o objetivo de aprimorar a comunicação entre dispositivos IoT e facilitar o desenvolvimento de soluções IoT mais inteligentes, adaptáveis e contextualmente relevantes. 

Palavras-chaves: Internet das Coisas (IoT). WoTPy. Grafos de Conhecimento. W3C-Web of Things (WoT).
\end{resumo}

% (pelo menos 3 e no máximo 5, lembrando que um conjunto de palavras pode formar o conceito que chamamos de palavras chave, por exemplo, "Sistemas de Informação", constitui UMA palavra chave).
