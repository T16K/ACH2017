% resumo em inglês
%-------------------------------------------------------------------------
% Comentário adicional do PPgSI - Informações sobre ``resumo em inglês''
% 
% Caso a Qualificação ou a Dissertação/Tese inteira seja elaborada no idioma inglês, 
% então o ``Abstract'' vem antes do ``Resumo''.
% 
%-------------------------------------------------------------------------
\begin{resumo}[Abstract]
\begin{otherlanguage*}{english}

%-------------------------------------------------------------------------
% Comentário adicional do PPgSI - Informações sobre ``referência em inglês''
% 
% Troque os seguintes campos pelos dados de sua Dissertação/Tese (mantendo a 
% formatação e pontuação):
%     - SURNAME
%     - FirstName1
%     - MiddleName1
%     - MiddleName2
%     - Work title: work subtitle
%     - DefenseYear (Ano de Defesa)
%
% Mantenha todas as demais informações exatamente como estão.
%
% [Não usar essas informações de ``referência'' para Qualificação]
%
%-------------------------------------------------------------------------
\begin{flushleft}
ARIGA, Gustavo Tsuyoshi. \textbf{Integration of WoTPy with Knowledge Graphs to Improve Interoperability in the Internet of Things (IoT).} \imprimirdata. \pageref{LastPage} p. Monograph – School of Arts, Sciences and Humanities, University of São Paulo, São Paulo, 2023. 
\end{flushleft}

In this work, the integration of WoTPy with knowledge graphs will be addressed, contributing to overcome the interoperability challenges in the growing field of the Internet of Things (IoT). WoTPy, a framework compliant to the W3C-Web of Things (WoT) standard, makes it easy do deploy IoT Gateways but faces significant challenges in terms of integrating and interpreting data from a variety of IoT devices. The focus of this project is to expand WoTPy's functionalities through integration with a knowledge graph architecture, providing a semantic interpretation of IoT data. By applying ontologies such as SSN (Semantic Sensor Network) and SOSA (Sensor, Observation, Sample, and Actuator), the study deepens the representation and interactions among IoT devices. Implementing an RDF/OWL model, along with a SPARQL endpoint, constitutes fundamental steps in this process. These implementations aim to enhance communication among IoT devices and facilitate the development of smarter, adaptable, and contextually relevant IoT solutions.

Keywords: Internet of Things (IoT). WoTPy. Knowledge Graphs. W3C-Web of Things (WoT).

\end{otherlanguage*}
\end{resumo}
