\chapter{Introdução}

A era digital contemporânea tem sido profundamente influenciada pelo avanço da Internet das Coisas (IoT). Este conceito, primeiro proposto por \cite{Ashton2009}, descreve um mundo onde objetos físicos estão interconectados através da Internet, trocando dados e criando uma realidade onde o digital e o físico coexistem. Esta interconexão abrange uma ampla variedade de dispositivos, desde sensores ambientais até eletrodomésticos inteligentes, todos interligados em uma rede digital complexa. A IoT revolucionou a maneira como interagimos com o ambiente ao nosso redor, trazendo inovações significativas para diversas áreas, incluindo saúde, indústria e urbanismo \cite{Atzori2010}.

O crescimento da IoT, conforme relatado por \cite{Gubbi2013}, é impulsionado por desenvolvimentos em tecnologias de conectividade, a redução de custos em sensores e dispositivos, e um aumento na capacidade de processamento e análise de dados. Este avanço resultou em aplicações que vão além do âmbito doméstico, impactando setores como a indústria 4.0, cidades inteligentes e saúde digital, transformando a forma como operações e decisões são realizadas nestes campos.

No entanto, este desenvolvimento acelerado também traz desafios consideráveis. Um dos mais críticos, conforme identificado por \cite{Vermesan2013}, é a interoperabilidade entre uma vasta gama de dispositivos e sistemas. Esta interoperabilidade é fundamental para realizar o potencial total da IoT, mas a falta dela pode resultar em silos de dados e limitações nas funcionalidades prometidas pela IoT.

Um desafio particularmente importante, como apontado por \cite{AlFuqaha2015}, é a interoperabilidade entre dispositivos de diferentes fabricantes. Esta questão é crucial devido à heterogeneidade dos dispositivos IoT, que variam em \textit{hardware}, sistemas operacionais, protocolos de comunicação e funcionalidades. A falta de interoperabilidade pode levar a falhas de comunicação entre dispositivos, criando ineficiências e potenciais riscos de segurança.

WoTPy é um arcabouço que implementa o padrão W3C para a Internet das Coisas, padrão também chamado \textit{Web of Things (WoT)}. O arcabouço facilita a comunicação entre dispositivos IoT mas enfrenta limitações na gestão e processamento de dados.

Este trabalho propõe a integração de um \textit{gateway IoT}, implementado usando WotPy, com grafos de conhecimento para enfrentar essas limitações. A integração com grafos de conhecimento visa aprimorar a interpretação semântica dos dados, para a eficácia dos sistemas IoT.

O objetivo principal deste trabalho é, portanto, expandir as funcionalidades do WoTPy, utilizando grafos de conhecimento para melhorar a interpretação dos dados na IoT. Ao abordar este problema, espera-se não apenas aumentar a eficiência e a eficácia dos sistemas IoT, mas também abrir caminhos para inovações futuras.

Este trabalho tem relevância prática e teórica para o campo da IoT, contribuindo para superar o desafio da interoperabilidade entre diferentes dispositivos de diferentes fabricantes. As contribuições esperadas incluem melhorias significativas na comunicação entre dispositivos IoT e avanços na interpretação semântica dos dados, capacitando sistemas IoT a tomar decisões automatizadas mais precisas e eficientes.
