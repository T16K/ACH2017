\chapter{Objetivos}

\section{Objetivo Geral}

O objetivo geral deste trabalho foi aprimorar as funcionalidades do WoTPy, integrando-o com grafos de conhecimento para melhorar a interoperabilidade e a interpretação semântica de dados na Internet das Coisas (IoT). Este objetivo envolveu o desenvolvimento de uma solução que não só facilita a comunicação entre dispositivos IoT de diferentes fabricantes, mas também aprimorou a capacidade de análise e interpretação de dados complexos, contribuindo para a evolução e eficiência dos sistemas IoT.

\section{Objetivos Específicos}

\begin{enumerate}
    \item Desenvolver e integrar funcionalidades no WoTPy que permitam um gerenciamento e processamento mais eficiente de dados de uma variedade maior de dispositivos IoT.
    \item Construir uma metodologia eficaz para integrar o WoTPy com grafos de conhecimento, utilizando as ontologias SSN (\textit{Semantic Sensor Network}) e SOSA (\textit{Sensor, Observation, Sample, and Actuator}) para estruturar os dados.
    \item Implementar mecanismos no WoTPy que permitam uma interpretação semântica aprofundada dos dados coletados, transformando dados brutos em informações contextualizadas.
    \item Criar um modelo RDF/OWL para representar os dados coletados de dispositivos IoT, facilitando a interoperabilidade e a análise semântica.
    \item Configurar um \textit{endpoint} SPARQL que permita consultas complexas e análise de dados dentro dos grafos de conhecimento, aumentando a acessibilidade e a usabilidade dos dados IoT.
    \item Conduzir testes para validar a eficácia da integração proposta, assegurando que as melhorias no WoTPy contribuam positivamente para a comunicação e análise de dados na IoT.
    \item Criar exemplos de uso e documentação para auxiliar programadores e usuários na implementação do WoTPy em seus projetos de IoT.
\end{enumerate}

Esses objetivos específicos visam abordar as lacunas identificadas nas capacidades atuais do WoTPy, elevando seu desempenho e funcionalidade para atender às demandas crescentes do ecossistema IoT.
