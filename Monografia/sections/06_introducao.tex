\chapter{Introdução}

\chapter{Relevância ou Justificativa}

% (Por que é importante realizar este trabalho? Quais são os benefícios e quem pode se beneficiar dos resultados deste projeto?)

A realização deste trabalho é de grande relevância devido à crescente demanda por soluções que facilitem a integração e interoperabilidade entre dispositivos IoT de diferentes fabricantes. À medida que a Internet das Coisas se expande e se torna cada vez mais comum, é fundamental abordar as dificuldades e limitações atuais de comunicação e integração entre dispositivos IoT como citado em \cite{Stirbu2008} \cite{Gyrard2017}. \cite{GARCIAMANGAS2019235}, \cite{OpenApíWoT2021}.

Ao fazer a contribuição para WoTPy, é possível oferecer um potencial de solução arquitetônica eficiente e padronizada para conectar dispositivos IoT, independentemente do fabricante, o que facilita a criação de aplicações e sistemas unificados. Essa padronização promovida pela Web das Coisas (WoT) também contribui para a democratização do acesso à tecnologia IoT, permitindo que desenvolvedores e usuários finais possam explorar e implementar soluções IoT de maneira mais eficiente e acessível.

Os benefícios esperados deste trabalho incluem:

\begin{itemize}
    \item Facilitar a instalação do WoTPy, tornando-o mais acessível para desenvolvedores e usuários finais;
    \item Fornecer documentação detalhada e exemplo prático para apoiar a adoção do WoTPy em projetos IoT.
\end{itemize}

Os possíveis beneficiários dos resultados deste projeto abrangem uma ampla gama de setores e atores, incluindo:

\begin{itemize}
    \item Fabricantes de dispositivos IoT que buscam simplificar a integração de seus produtos com outros dispositivos e sistemas;
    \item Desenvolvedores de software e engenheiros de sistemas interessados em construir soluções IoT eficientes e interoperáveis;
    \item Empresas e organizações que buscam implementar soluções IoT em seus processos e infraestruturas;
    \item Pesquisadores e acadêmicos que estudam e desenvolvem novas abordagens para a Internet das Coisas e a Web das Coisas.
\end{itemize}

A integração do WoTPy com grafos de conhecimento tem como objetivo melhorar ainda mais a interoperabilidade entre dispositivos IoT de diferentes fabricantes, permitindo uma comunicação e integração mais eficiente e semântica entre eles.

A área da SWOT \cite{Scioscia2009} \cite{Jara2014SWoT} é baseada em conceitos que visam aproveitar o potencial da Web Semântica para melhorar a interação entre dispositivos IoT. Ao adotar os princípios e tecnologias da Web Semântica, é possível criar uma camada de conhecimento compartilhado e representações semânticas das informações dos dispositivos IoT \cite{bernerslee2001semantic}. Essa abordagem facilita a busca, integração e análise de dados, permitindo a criação de aplicações e serviços mais inteligentes e personalizados.

