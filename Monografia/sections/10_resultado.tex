\chapter{Resultados}

Os resultados esperados decorrentes do trabalho a ser desenvolvido abrangem várias áreas relacionadas à implementação e melhoria do WoTPy como um gateway experimental baseado no W3C-WoT. Estes resultados estão alinhados aos objetivos específicos do trabalho e visam impactar positivamente a interoperabilidade e integração de dispositivos IoT de diferentes fabricantes. A seguir, são apresentados os principais resultados esperados:

\begin{enumerate}
    \item Compreensão detalhada das especificações do W3C-WoT, possibilitando a implementação correta e eficiente do WoTPy de acordo com os padrões e diretrizes estabelecidos pelo W3C;
    \item Seleção e domínio das bibliotecas e ferramentas adequadas para o desenvolvimento do WoTPy, garantindo o suporte aos protocolos de comunicação necessários e a facilidade de integração com outros projetos IoT;
    \item Resolução dos problemas de instalação do WoTPy, tornando-o facilmente implantável e configurável em diferentes ambientes e sistemas operacionais, o que facilita a adoção do projeto por desenvolvedores e usuários finais;
    \item Desenvolvimento de exemplos de uso e documentação detalhada para auxiliar outros desenvolvedores e usuários na implementação do WoTPy em seus próprios projetos IoT, promovendo a disseminação e adoção do projeto pela comunidade;
    \item Validação das contribuições realizadas por meio do conjunto de testes do WoTPy.
\end{enumerate}

Com a obtenção desses resultados, espera-se que o WoTPy se torne um gateway funcional e eficaz, capaz de facilitar a comunicação e integração entre dispositivos IoT de diferentes fabricantes. Além disso, espera-se que o trabalho contribua para a popularização do WoTPy e do conceito de Web das Coisas, melhorando a interoperabilidade e a eficiência dos sistemas IoT em geral.

\section{Especificações do W3C-WoT}

O Web of Things (WoT) é uma iniciativa do World Wide Web Consortium (W3C) que busca estabelecer uma estrutura para conectar dispositivos e serviços na Internet das Coisas (IoT). O objetivo do WoT é permitir a interoperabilidade entre esses dispositivos, facilitando a comunicação e a integração padronizadas.

O W3C desenvolveu várias especificações que abordam diferentes aspectos do WoT. Essas especificações são fundamentais para a implementação e adoção do WoT, proporcionando diretrizes para descrever as características das Things, definir interfaces de rede, realizar a descoberta de Things, implementar a lógica de aplicação e garantir a segurança e a privacidade dos sistemas WoT.

\begin{itemize}
   \item WoT Thing Description \citeonline{TD} define um formato de dados legível por máquina para descrever os metadados e as interfaces de rede das Things. Ela fornece uma base sólida para a interoperabilidade entre as Things e a Web.
   \item WoT Binding Templates \citeonline{WoTBinding} oferecem orientações sobre como definir interfaces de rede em Things para protocolos específicos e ecossistemas de IoT. Esses modelos são úteis para garantir a compatibilidade e a integração de diferentes sistemas.
    \item WoT Discovery \citeonline{WoTDiscovery} define um mecanismo de distribuição de metadados das Things. Ela permite a localização e o acesso a informações detalhadas sobre as Things, facilitando a descoberta e a integração de dispositivos na rede.
    \item WoT Scripting API \citeonline{WoTScripting} possibilita a implementação da lógica de aplicação das Things utilizando uma API JavaScript comum. Isso simplifica o desenvolvimento de aplicativos IoT e promove a portabilidade entre fornecedores e dispositivos.
    \item WoT Security and Privacy Guidelines \citeonline{WoTSecurity} oferecem orientações para a implementação segura das Things e discutem questões relacionadas à segurança e à privacidade nos sistemas WoT. Essas diretrizes são importantes para proteger os dispositivos e os dados sensíveis envolvidos nas redes WoT.
\end{itemize}

Essas especificações do W3C-WoT desempenham um papel fundamental na construção de um ecossistema interoperável e seguro para a Internet das Coisas, permitindo a integração harmoniosa de dispositivos e serviços em diferentes domínios.

\section{Dependências do W3C-WoTPy}

O arquivo ''setup.py'' \citeonline{gitwotpy:setup} é responsável por definir as dependências necessárias para a execução da biblioteca WotPy, conforme especificado pelo World Wide Web Consortium (W3C) \cite{WoTArchitecture}. Essas dependências podem ser agrupadas em duas categorias: as obrigatórias e as opcionais.

Entre as principais dependências obrigatórias definidas no arquivo setup.py, estão:

\begin{itemize}
    \item tornado: uma biblioteca assíncrona utilizada para criar aplicativos web em Python. É empregada pela WotPy para estabelecer servidores HTTP e WebSocket.
    \item jsonschema: uma biblioteca que valida esquemas JSON e os dados JSON correspondentes. A WotPy utiliza essa biblioteca para validar as descrições de coisa (Thing Descriptions) recebidas e geradas.
    \item six: uma biblioteca que possibilita escrever código Python compatível tanto com a versão 2 quanto com a versão 3. A WotPy se beneficia dessa biblioteca para garantir a compatibilidade entre as duas versões do Python.
    \item rx: uma biblioteca de programação reativa que permite escrever código que responde assincronamente a mudanças de estado. É utilizada pela WotPy para suportar a API WoT Scripting e as interações com propriedades e eventos observáveis.
    \item python-slugify: uma biblioteca que converte strings para "slug", um formato de texto que utiliza apenas caracteres ASCII, números e traços. A WotPy utiliza essa biblioteca para criar identificadores únicos para as coisas.
\end{itemize}

Além dessas dependências obrigatórias, existem algumas dependências opcionais que a WotPy utiliza se estiverem disponíveis no sistema:

\begin{itemize}
    \item aiocoap: uma biblioteca Python para o protocolo de transferência de dados Constrained Application Protocol (CoAP), utilizada pela WotPy para suportar o protocolo CoAP.
    \item hbmqtt: uma biblioteca Python para implementar o protocolo Message Queue Telemetry Transport (MQTT), utilizada pela WotPy para suportar o protocolo MQTT.
    \item websockets: uma biblioteca Python para suportar a comunicação via WebSocket, utilizada pela WotPy para suportar o protocolo WebSocket.
    \item zeroconf: uma biblioteca Python para suportar o protocolo DNS Service Discovery (DNS-SD), utilizada pela WotPy para descobrir serviços e dispositivos na rede.
\end{itemize}

Essas dependências são verificadas em tempo de execução e adicionadas ao conjunto de dependências da biblioteca WotPy, caso estejam disponíveis no sistema.

\section{Problemas de Instalação}

Este relatório aborda os problemas encontrados durante a execução e construção do projeto WotPy, disponível no GitHub \cite{gitwotpy:2022}. Inicialmente, foi realizado um fork do projeto WotPy e o repositório \citeonline{gitwotpy:2023} foi clonado para o computador. 

\subsection{Construção do Projeto com Docker}

Ao tentar construir o projeto utilizando o \textit{docker build .}, deparou-se com um erro relacionado à versão do pacote numpy. A mensagem de erro está descrito nesse arquivo \cite{gitwotpy:cpd}.

Para solucionar esse problema, foi tentada inicialmente a alteração da versão do Python no arquivo ''Dockerfile'' \cite{gitwotpy:v1}.

Posteriormente, optou-se por utilizar a versão original do Python (3.7) e, no arquivo ''examples/benchmark/requirements.txt'', foi revertida a versão modificada pelo dependabot[bot] \cite{gitwotpy:bot} \cite{gitwotpy:v2}.

Por fim, a solução adotada foi remover a inclusão das dependências do diretório ''/examples/benchmark'' no ''Dockerfile'', mantendo a modificação anterior \cite{gitwotpy:v5}. Essa decisão foi tomada considerando que o exemplo do benchmark não seria utilizado no projeto, e, portanto, as dependências relacionadas a ele não seriam necessárias.

\subsection{Execução dos Testes}

A execução dos testes do WoTPy foi iniciada com o seguinte comando\textit{./pytest-docker-all.sh}. Após a construção, foi identificado o erro descrito nesse arquivo \cite{gitwotpy:et}. Para lidar com esse problema, o arquivo setup.py foi editado \cite{gitwotpy:v1}.

Dessa forma, o WoTPy pôde ser construído corretamente utilizando o Docker e passou nos testes propostos no "pytest-docker-all.sh".

\section{Desenvolvimento do Exemplo de Uso}

Este projeto demonstra a transmissão de dados do sensor UV de um microcontrolador ESP32 para um servidor Web of Things (WoT) via protocolo HTTP. O projeto utiliza a biblioteca WotPy para criação do servidor e interações com o WoT, e o microcontrolador ESP32 emparelhado com um sensor UV ML8511.

Dois principais arquivos de código compõem este projeto: server.py e main.py. O arquivo server.py configura o servidor WoT, expõe um Thing que representa o sensor UV, e define handlers customizados para leitura e gravação dos dados do sensor UV. O main.py, rodando no ESP32, periodicamente lê os dados do sensor UV e os envia para o servidor WoT usando requisições HTTP.

Este projeto serve como um exemplo base de integração de dispositivos IoT com princípios do WoT, facilitando a comunicação e a interoperabilidade entre dispositivos e aplicações dentro de um ecossistema descentralizado de IoT.

\subsection{Processo de Comunicação}

A situação envolve um dispositivo ESP32 equipado com um sensor ML8511 (sensor UV) atuando como cliente, enquanto um computador atua como servidor utilizando a biblioteca WoTPy.

\subsubsection{Comunicação entre o Cliente ESP32 e o Servidor WoT utilizando o Protocolo HTTP}

A comunicação entre o cliente ESP32 e o servidor Web of Things (WoT) é estabelecida por meio do protocolo HTTP, sendo uma escolha adequada para esse contexto. O uso do protocolo HTTP apresenta diversas vantagens que tornam essa abordagem viável e eficiente para o projeto em questão.

Uma das razões para a escolha do protocolo HTTP é a minha familiaridade com essa tecnologia. Como pesquisador e desenvolvedor, possuo conhecimento aprofundado sobre as especificações do protocolo e me sinto confortável em implementar a comunicação utilizando-o. Essa familiaridade agiliza o processo de desenvolvimento, evitando a necessidade de aprender um novo protocolo e me permitindo aproveitar minha experiência prévia nessa área.

Além disso, o HTTP é amplamente utilizado na web e conta com recursos de implementação disponíveis para diferentes plataformas e linguagens de programação, incluindo o MicroPython utilizado no ESP32. Existem bibliotecas e ferramentas prontamente disponíveis que facilitam a implementação da comunicação HTTP, possibilitando um desenvolvimento mais eficiente e eficaz.

Outra vantagem do protocolo HTTP é a sua simplicidade de implementação. O protocolo possui uma estrutura bem definida e é relativamente fácil de entender e utilizar. Isso simplifica o processo de comunicação entre o ESP32 e o servidor WoT, tornando a implementação mais acessível e reduzindo a possibilidade de erros.

O protocolo HTTP é altamente compatível com a infraestrutura existente na web. Roteadores, firewalls e servidores HTTP são projetados para suportar e lidar com o tráfego HTTP de forma eficiente. Ao utilizar o HTTP para a comunicação do ESP32 com o servidor WoT, é possível aproveitar essa infraestrutura existente sem a necessidade de configurações complexas adicionais.

O HTTP também oferece suporte a uma variedade de métodos de solicitação, como GET, POST, PUT e DELETE, permitindo que o cliente ESP32 envie solicitações para ler, gravar e excluir dados no servidor WoT. Essa flexibilidade é fundamental para a interação entre o cliente e o servidor, possibilitando a troca de informações necessárias para o correto funcionamento do sistema WoT.

Dessa forma, a escolha do protocolo HTTP para a comunicação entre o cliente ESP32 e o servidor WoT se mostra adequada, considerando não apenas a minha familiaridade com esse protocolo, mas também sua ampla adoção, simplicidade de implementação, compatibilidade com a infraestrutura existente, suporte a diferentes métodos de solicitação e recursos de segurança. Essa abordagem proporciona uma comunicação eficiente, confiável e segura entre os dispositivos, viabilizando a troca de dados e o correto funcionamento do sistema WoT.

\subsubsection{Gateway na Arquitetura da Internet das Coisas (IoT)}

Na arquitetura da Internet das Coisas (IoT), um gateway desempenha um papel fundamental como ponto de conexão entre a nuvem (ou servidor) e os dispositivos, sensores e atuadores no campo. No contexto do projeto WoT com WoTPy, o gateway possui diversas funções essenciais que contribuem para o correto funcionamento e gerenciamento da rede IoT.

Um dos principais papéis do gateway é atuar como um tradutor de protocolos, facilitando a comunicação entre dispositivos que utilizam protocolos de comunicação diferentes. Por exemplo, no caso do projeto com ESP32 e sensor ML8511, onde o dispositivo ESP32 pode utilizar o protocolo HTTP ou MQTT, enquanto outros dispositivos na rede podem utilizar CoAP, WebSocket ou outros protocolos. O gateway WoT é capaz de traduzir entre esses protocolos, garantindo a interoperabilidade e permitindo a comunicação harmoniosa entre os dispositivos.

Além da tradução de protocolos, o gateway também desempenha outras funções importantes. Ele pode realizar a agregação e pré-processamento de dados provenientes de múltiplos dispositivos antes de enviá-los para a nuvem. Essa capacidade permite a combinação de dados de diferentes sensores, a realização de cálculos básicos nos dados ou a redução da quantidade de dados a serem enviados para a nuvem, otimizando assim o uso da largura de banda e os recursos de armazenamento.

Outra função relevante do gateway é o gerenciamento de dispositivos. O gateway oferece recursos de configuração, atualização de firmware e monitoramento do estado dos dispositivos IoT. Isso inclui a possibilidade de configurar parâmetros dos dispositivos, como frequência de amostragem, intervalo de transmissão e outros aspectos relacionados ao seu funcionamento.

A segurança também é uma preocupação importante no contexto da IoT, e o gateway desempenha um papel crucial nesse aspecto. Ele pode fornecer funcionalidades de autenticação e autorização de dispositivos, criptografia de dados e proteção contra ameaças de segurança, garantindo assim a integridade, confidencialidade e disponibilidade dos dados transmitidos na rede IoT.

Por fim, alguns gateways têm a capacidade de realizar computação de borda (edge computing), processando dados localmente em vez de enviá-los para a nuvem. Essa abordagem reduz a latência na resposta aos dados coletados, preserva a privacidade dos dados e melhora a eficiência do uso da largura de banda, sendo especialmente útil em cenários em que a conectividade com a nuvem é limitada ou a quantidade de dados a serem transmitidos é muito grande.

No contexto do sistema com o ESP32 e o sensor ML8511, o gateway (representado pelo computador) recebe os dados do sensor ML8511 por meio do ESP32, expondo esses dados na Web of Things e fornecendo um ponto de acesso para que outros clientes WoT possam acessá-los. O gateway desempenha um papel crucial na tradução de protocolos, agregação de dados, gerenciamento de dispositivos e segurança, garantindo o correto funcionamento e a interoperabilidade na rede IoT.

\subsubsection{Thing Description na Arquitetura da Web of Things (WoT)}

A "Thing Description" (TD) é um elemento fundamental na arquitetura da Web of Things (WoT). Ela consiste em uma representação de alto nível do dispositivo ou "Thing" na Web of Things, fornecendo um conjunto de metadados que descrevem o dispositivo e suas capacidades em termos de propriedades, ações e eventos.

A seção de metadados da Thing Description inclui informações gerais sobre o dispositivo, como seu nome, tipo e outras informações relevantes para clientes e dispositivos na rede. Essas informações ajudam a identificar e contextualizar o dispositivo na rede IoT, facilitando sua descoberta e utilização.

As propriedades representam o estado atual do dispositivo, permitindo que os clientes obtenham informações sobre o dispositivo em tempo real. Por exemplo, no caso de um sensor de luz, uma propriedade pode ser o valor atual da luz detectada. As ações representam as funcionalidades que podem ser executadas no dispositivo, permitindo que os clientes interajam com ele. Por exemplo, um dispositivo de luz inteligente pode ter ações como "ligar" e "desligar". Os eventos representam notificações ou alertas que o dispositivo pode emitir, permitindo que os clientes sejam informados sobre mudanças ou condições específicas. Por exemplo, um sensor de temperatura pode enviar um evento quando a temperatura ultrapassa um determinado limite.

A Thing Description é formatada como um documento JSON-LD, tornando-a facilmente legível tanto para humanos quanto para máquinas. Essa formatação também permite que a TD seja incorporada em uma variedade de sistemas e plataformas, promovendo a interoperabilidade entre dispositivos IoT independentemente do protocolo de rede ou tecnologia subjacente utilizada.

No contexto do sistema utilizando o ESP32 e o sensor ML8511, seria criada uma Thing Description para representar o sensor ML8511, incluindo metadados sobre o sensor e descrevendo suas propriedades (como o valor atual de UV), ações disponíveis (se houver) e eventos que o sensor pode emitir. Essa descrição padronizada permite que outros dispositivos ou clientes WoT interajam com o sensor de forma consistente e interoperável, facilitando a troca de informações e a integração na Web of Things.

\subsection{Cliente Web of Thing (WoT)}

O código ''main.py'' \citeonline{gitwotpy:main} implementa um cliente Web of Things (WoT) no dispositivo ESP32, permitindo a interação com sensores e o envio dos dados para um servidor WoT. O cliente WoT é responsável por buscar informações dos sensores e controlar as funcionalidades dos dispositivos conectados de maneira padronizada e interoperável.

O cliente WoT utiliza o protocolo HTTP para se comunicar com o servidor WoT. No código, é feito uso da biblioteca \textit{urequests} para realizar solicitações HTTP PUT ao servidor. Essas solicitações enviam os dados dos sensores em formato JSON para o servidor WoT, permitindo que os dados sejam armazenados e acessados pelos usuários.

O código define os sensores utilizados no dispositivo ESP32, como o sensor de UV, e suas configurações. Cada sensor possui uma identificação única e é associado a uma descrição no formato de Thing Description (TD). A TD contém informações sobre o sensor, como os links de acesso aos dados no servidor WoT.

A função send_sensor_data é responsável por enviar os dados dos sensores para o servidor WoT. Ela recebe como parâmetros a identificação do sensor, o tipo de sensor, o URL de acesso aos dados na TD e os dados a serem enviados. A função realiza uma solicitação HTTP PUT ao URL especificado, enviando os dados em formato JSON. Se a solicitação for bem-sucedida, uma mensagem de sucesso é exibida. Caso contrário, uma mensagem de falha é exibida juntamente com o código de status da resposta.

No loop principal main(), os valores dos sensores são lidos periodicamente. A cada iteração do loop, os dados dos sensores são obtidos e enviados para o servidor WoT utilizando a função send_sensor_data. Esse processo permite que os dados dos sensores sejam atualizados no servidor WoT em tempo real, possibilitando que os usuários acessem e utilizem essas informações de forma padronizada.

Em resumo, o código implementa um cliente WoT no ESP32, permitindo a comunicação com sensores e o envio dos dados para um servidor WoT por meio do protocolo HTTP. O uso da Thing Description (TD) possibilita a descrição dos dispositivos e a padronização das interações com o servidor. Essa abordagem permite que os usuários acessem os dados dos sensores de forma fácil e interoperável, promovendo a criação de aplicações e serviços baseados na Web of Things.

\subsection{Servidor Web of Things (WoT)}

O servidor Web of Things (WoT) é uma peça fundamental na arquitetura WoT, pois é responsável por expor os dispositivos conectados e suas funcionalidades para que possam ser descobertos e interagidos pelos clientes. O servidor WoT atua como uma ponte entre os dispositivos da Internet das Coisas (IoT) e os aplicativos e serviços que desejam acessá-los.

O código server.py \citeonline{gitwotpy:server} implementa um servidor Web of Things (WoT) que expõe uma Thing (dispositivo) responsável por fornecer os valores de um sensor de UV. O servidor WoT permite a descoberta e interação com a Thing por meio de solicitações HTTP padronizadas.

O servidor WoT utiliza a biblioteca Tornado e o protocolo HTTP para receber solicitações dos clientes e responder de acordo com as interações definidas na descrição da Thing. A Thing é definida por uma descrição no formato de um documento JSON, que especifica suas propriedades e comportamentos.

No código, é criado um HTTP server na porta especificada (HTTP_PORT) e um Servient, que é uma instância responsável por gerenciar os recursos WoT. A descrição da Thing é definida, contendo o identificador (ID_THING) e as propriedades, como o sensor de UV.

Para cada propriedade da Thing, como o sensor de UV, são definidos os manipuladores de leitura (read_uv) e escrita (write_uv). O manipulador de leitura é responsável por retornar o valor atual do sensor quando solicitado pelo cliente. O manipulador de escrita é responsável por atualizar o valor do sensor quando recebido uma solicitação de escrita do cliente.

Ao iniciar o servidor, é criado o objeto WoT, e a Thing é produzida com base na descrição. Em seguida, os manipuladores de leitura e escrita são associados à propriedade correspondente na Thing. Isso permite que o servidor WoT responda às solicitações de leitura e escrita para o sensor de UV.

Por fim, a Thing é exposta e o servidor inicia seu loop de eventos, aguardando as solicitações dos clientes e respondendo de acordo com as interações definidas.

Em resumo, o código implementa um servidor WoT que expõe uma Thing representando um sensor de UV. O servidor utiliza o protocolo HTTP e a biblioteca Tornado para receber e responder solicitações dos clientes. A descrição da Thing é definida, e manipuladores de leitura e escrita são associados às propriedades da Thing. Isso permite que os clientes leiam e escrevam dados no servidor WoT de forma padronizada e interoperável.

