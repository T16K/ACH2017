% ------------------------------------------------------------------------
% ------------------------------------------------------------------------
% abnTeX2: Modelo de Trabalho Academico (tese de doutorado, dissertacao de
% mestrado e trabalhos monograficos em geral) em conformidade com 
% ABNT NBR 14724:2011: Informacao e documentacao - Trabalhos academicos -
% Apresentacao
%
% baseado no modelo "Template para elaboração de trabalho acadêmico - versão setembro/2016" fornecido pela Biblioteca do IFSC - http://www.ifsc.edu.br/menu-colecao-abnt (acessado em 2017-08-23)
% ------------------------------------------------------------------------
% ------------------------------------------------------------------------
\documentclass[
	% -- opções da classe memoir --
	10pt,				% tamanho da fonte
	oneside,			% para impressão só frente
	a4paper,			% tamanho do papel. 
	% -- opções da classe abntex2 --
	chapter=TITLE,		% títulos de capítulos convertidos em letras maiúsculas
	%section=TITLE,		% títulos de seções convertidos em letras maiúsculas
	%subsection=TITLE,	% títulos de subseções convertidos em letras maiúsculas
	%subsubsection=TITLE,% títulos de subsubseções convertidos em letras maiúsculas
	% -- opções do pacote babel --
	english,			% idioma adicional para hifenização
	brazil				% o último idioma é o principal do documento
	]{abntex2}

%	Todas as indicações de pacotes e configurações estão no arquivo de estilo
%  chamado estilo-monografia-ifsp.sty.
\usepackage{preamble}	

% \documentclass[tikz]{standalone}
\usepackage{pgfgantt} % https://pt.overleaf.com/latex/examples/gantt-charts-with-the-pgfgantt-package/jmkwfxrnfxnw
% \title{Gantt Charts with the pgfgantt Package}
% \begin{document}

\usepackage{xcolor} % text colors https://www.overleaf.com/learn/latex/Using_colours_in_LaTeX

%---------------------------------------------------------------------%
%---------------------------------------------------------------------%
% Informações de dados para CAPA e FOLHA DE ROSTO
%---------------------------------------------------------------------%
%---------------------------------------------------------------------%

\titulo{WotPy: Solução de Problemas e Exemplo de Uso}
\autor{Gustavo Tsuyoshi Ariga}
\local{São Paulo, SP}
\data{2023}
\orientador{Professores Fábio Nakano e José de Jesús Pérez-Alcázar}
\instituicao{%
  Universidade de São Paulo - USP
  \par
  Escola de Artes, Ciências e Humanidades
  \par
  Graduação em Sistemas de Informação}
\tipotrabalho{Plano de Atividades}

% O preambulo deve conter o tipo do trabalho, o objetivo, 
% o nome da instituição e a área de concentração 
\preambulo{Plano de atividades apresentado como parte dos requisitos necessários para cumprimento da disciplina ACH2017 ou ACH2018 - Projeto Supervisionado ou de Graduação I ou II.}
%---------------------------------------------------------------------%
%\textoaprovacao{}


%---------------------------------------------------------------------%
% Início do documento
%---------------------------------------------------------------------%


\begin{document}
% Seleciona o idioma do documento (conforme pacotes do babel)
\selectlanguage{brazil}
% Retira espaço extra obsoleto entre as frases.
\frenchspacing 


% ----------------------------------------------------------
% ELEMENTOS PRÉ-TEXTUAIS
% ----------------------------------------------------------
% \pretextual

\imprimircapa
% Folha de rosto - (o * indica que haverá a ficha bibliográfica)
\imprimirfolhaderosto*
% ---
%---------------------------------------------------------------------%


% ----------------------------------------------------------
% Inclusão dos capítulos que estão em outros arquivos .tex
% ----------------------------------------------------------
\chapter{Resumo}

% (Resumo do trabalho.)

% Introdução
À medida que a Internet das Coisas (IoT) se populariza, cresce também a demanda por sistemas capazes de integrar dispositivos de várias marcas em uma aplicação unificada. No entanto, as restrições atuais da IoT tornam difícil a interoperabilidade entre esses dispositivos. Para contornar esse problema, o consórcio World Wide Web (W3C) \cite{W3CAbout} sugeriu a Web das Coisas (WoT) \cite{WoTCommunityWiki} como uma solução arquitetônica que utiliza a Web para promover a comunicação eficiente entre dispositivos IoT.

% Objetivos
Dentro desse cenário, o presente trabalho tem como objetivo implantar e aprimorar o WoTPy \cite{GARCIAMANGAS2019235}, um gateway experimental baseado no W3C-WoT \cite{WoTArchitecture}, compatível com protocolos como HTTP, Websockets, MQTT e CoAP. O plano de atividades abrange a solução dos problemas de instalação do WoTPy \url{https://github.com/agmangas/wot-py}, a documentação do processo e a elaboração de exemplo prático de uso. Adicionalmente, caso tudo ocorra como planejado, no próximo semestre (segundo semestre de 2023), será abordada a integração do WoTPy com grafos de conhecimento, visando melhorar a interoperabilidade dos dispositivos IoT e possibilitar a divulgação das capacidades dos sensores e suas observações através de um endpoint SPARQL \url{https://sparqlwrapper.readthedocs.io/en/latest/}.

% Conclusão
Ao longo do projeto, serão investigados os materiais e métodos necessários para alcançar os objetivos estabelecidos, como a seleção de bibliotecas adequadas e das dependências. O resultado esperado é um gateway WoTPy funcional, que possa ser facilmente instalado e utilizado para aprimorar a integração e comunicação entre dispositivos IoT de diferentes fabricantes.


\chapter{Palavras Chaves}

% (Lista de palavras-chaves.)

Internet das Coisas (IoT), Web das Coisas (WoT), World Wide Web Consortium (W3C), WoTPy


\chapter{Modalidade}

\begin{description}
	\item (  ) Trabalho de Graduação Curto – 1 semestre - individual
	\item (X) Trabalho de Graduação Longo (parte 1) – 1 ano – individual
	\item (  ) Trabalho de Graduação Longo (parte 2) – 1 ano - individual			
	\item (  ) Trabalho de Graduação Curto – 1 semestre – grupo
	\item (  ) Trabalho de Graduação Longo (parte 1) – 1 ano – grupo
	\item (  ) Trabalho de Graduação Longo (parte 2) – 1 ano - grupo
\end{description}


\chapter{Apresentação do Problema}

% (Texto descrevendo o contexto e o problema abordado no TCC.)

% Interesse dos orientadores
Há alguns anos os orientadores colaboram, atuando com dispositivos (computacionais) físicos, Internet e Web Semântica, tendo orientado trabalhos de Iniciação Científica e de Conclusão de Curso. Do ponto de vista dos orientadores, que receberam um orientando interessado em desenvolver trabalho neste assunto, o ponto passa a ser \textit{onde aplicar a capacidade de trabalho do orientando em assuntos de interesse dos orientadores de maneira a dar alguma contribuição no assunto}. 

% Referências a partir das quais o texto é escrito
Com a intenção de fornecer referências fixas para elaboração deste plano de atividades, algumas escolhas de termos e conceitos serão feitas sem considerações detalhadas que, acredita-se, não são pertinentes ao escopo deste plano de atividades.

Web das Coisas (\textit{Web of Things}) é um termo criado por volta de 2007 e tem várias interpretações. Atualmente é definido pelo World Wide Web Consortium (W3C) como:

\textit{The Web of Things includes sensors and actuators, physical objects and locations, and even people. The Web of Things is essentially about the role of Web technologies to facilitate the development of applications and services for things and their virtual representation. Some relevant Web technologies include HTTP for accessing RESTful services, and for naming objects as a basis for linked data and rich descriptions, and JavaScript APIs for virtual objects acting as proxies for real-world objects. Key benefits for doing this work at W3C include the W3C emphasis on ensuring that W3C standards can be implemented royalty free, thereby encouraging innovation, and the availability of the large community of web developers. Standards are needed to realize the economic and human potential, and to avoid the risk of fragmentation cased by a plethora of non-interoperable proprietary solutions.
} \cite{WoTTerminology}, \cite{WoTCommunityWiki}

Em tradução livre, \textit{a Web das Coisas inclui sensores e atuadores, objetos físicos e locais e mesmo pessoas. A Web das Coisas é essencialmente sobre o papel das tecnologias Web na facilitação do desenvolvimento de aplicações e serviços para coisas e sua representação (no ambiente) virtual. Algumas tecnologias Web relevantes incluem HTTP para acessar serviços RESTful e para nomear objetos, como uma base para dados ligados e descrições (semanticamente) ricas e interfaces JavaScript para objetos virtuais que agem como representações de objetos do mundo real. Os benefícios deste trabalho realizado pela W3C incluem a ênfase que as normas possam ser implementadas sem necessidade de pagamento de royalties, portanto, encorajando inovação, e a disponibilidade de uma grande comunidade de desenvolvedores web. Normas são necessárias para viabilizar o potencial econômico e humano e evitar o risco de fragmentação causado pela proliferação de soluções proprietárias não interoperáveis}.

A escolha dessa definição para Web das Coisas tem como consequência a adesão, no mínimo parcial, a um conjunto de padrões criados pelo W3C, então cabe analisar se esta escolha é conveniente.

Nessa definição está implícito o desejo de interconectar sensores, atuadores, objetos físicos, locais e pessoas através de uma rede (um sistema de informação). É possível especular sobre consequências da implantação da tal rede, por exemplo seu impacto econômico, como fez a consultoria McKinsey em 2015 \cite{manyika2015}. 

Considerando os desafios técnicos e os interesses econômicos, é de se esperar que, em contraposição à formação de um consórcio para chegar a uma solução (tecnológica) consensual, diferentes organizações criarão (como criaram) suas próprias redes, abordando casos (exemplos, testes) que têm especificidades, propondo soluções particulares a problemas. Fazer essas redes interoperar, em diversos níveis, como compartilhar observações de sensores, compartilhar sensores, séries históricas de dados, análises de séries históricas, ... é um grande desafio que é citado em \cite{Stirbu2008} \cite{Gyrard2017}. \cite{GARCIAMANGAS2019235}, \cite{OpenApíWoT2021}. A persistência desse desafio ao longo do tempo indica que, apesar das várias propostas de soluções e ``normas'', não há consenso em todos os aspectos.

Neste cenário, nota-se que o W3C declara ser uma comunidade internacional de organizações-membro. A missão do W3C é desenvolver o potencial da World Wide Web (WWW) através da criação de normas através de um processo baseado em consenso e alimentado pelas contribuições de seus membros. (Composição da informação em \cite{W3CAbout}, \cite{W3CFacts}, \cite{W3CMission}). Várias empresas e pesquisadores (citados nos artigos acima) contribuem ou contribuíram para o W3C.

É possível confiar que as propostas, recomendações e normas do W3C contenham elementos que já estão em uso, tenham boa abrangência, devido ao processo de criação, e que persistam com o passar do tempo, apesar da certeza que as tecnologias, como HTTP, REST, JavaScript, tornar-se-ão obsoletas em algum momento. 

As atividades do W3C em torno da Web das Coisas definem, até certo ponto, a arquitetura para a Web das Coisas  \cite{Matsukura:23:WTA} e já conta com ferramentas que usam e implementam essa arquitetura \cite{WoTDevTools}.

Considerando que o argumento acima indique grande uso, abrangência, persistência, informação, documentação e quantidadade de ferramentas disponíveis, conclui-se que convém seguir, mesmo que parcialmente, o conceito de Web das Coisas proposto pelo W3C.\footnote{Neste ponto do texto, acredita-se que esteja claro que interoperabilidade entre redes IoT ainda é um desafio, que interoperabilidade pode se dar em vários níveis e que é conveniente seguir as recomendações do W3C.}
\footnote{Interoperabilidade, como tema amplo, é citado como um dos grandes desafios na área de Sistemas de Informação \cite{grandsi}} Cabe agora estreitar o escopo do trabalho para chegar ao objetivo.

% Parti de um conceito, cheguei a um conjunto de recomendações...

% Por outro lado, todos autores dos artigos citados no parágrafo anterior concordam que deve existir um ou mais agentes encarregados de traduzir protocolos e, desta forma, implementar interoperabilidade. Esses agentes, geralmente, são designados \textit{gateways}.

% Outros aspectos como segurança, privacidade, ... também estão presentes



% como fez Tim Berners-Lee, quando propôs a Web Semântica \cite{bernerslee2001semantic}

A arquitetura para Web das Coisas \cite{Matsukura:23:WTA} destaca os elementos que fazem parte da rede, e sua interconexão. Tal arquitetura pode ser detalhada em vários sentidos, por exemplo, definindo como esses elementos são organizados e, caso sejam dispositivos computacionais, o que é executado neles. Entretanto, aparentemente, o foco do W3C, está no (re)uso da infraestrutura e dos protocolos criados para Internet. Consequentemente, a coleção de exemplos de aplicação da Web das Coisas, apresentada na seção 4 e os padrões (\textit{pattern}) usados para implantação trazem poucos detalhes, mas o protocolo de comunicação entre os dispositivos é detalhado até o conteúdo, codificado no protocolo JSON, da requisição POST, codificada no protocolo HTTP, trocada entre dispositivos \cite{McCool:23:WTT}.

Nota-se que, exceto em aplicações muito simples, a arquitetura prevê a existência de dispositivos na fronteira entre a rede interna e a rede externa (Internet), região frequentemente referenciada como borda (\textit{edge}) \cite{EdgeComputing}. A computação realizada em dispositivos de borda frequentemente é denominada \textit{fog computing} \cite{FogComputing}. Pela proximidade com os sensores e atuadores, que frequentemente são limitados em capacidade de computação, disponibilidade de energia, tamanho, acesso físico direto, entre outras, a maioria das propostas e implementações, quando preocupam-se com interoperabilidade entre dispositivos, implementam esta característica no dispositivo de borda \cite{Stirbu2008} \cite{Gyrard2017}. \cite{GARCIAMANGAS2019235} que frequentemente é denominado \textit{gateway}. 

Por sua posição na rede, o \textit{gateway} é o dispositivo onde a maioria dos desenvolvedores implementa, além da interoperabilidade entre dispositivos, privacidade e segurança de dados, descoberta de dispositivos, interfaces (básicas) com o usuário. Desta forma, o \textit{gateway} é uma plataforma necessária, talvez essencial, para a construção de serviços baseados em dispositivos e nos dados gerados por estes.

% Um projeto que teve relativo sucesso é Mozilla Web of Things \cite{MozillaIoT}. Um artigo escrito no escopo do projeto Mozilla Web of Things \cite{MozillaIoTPrivacy} bem a preocupação com privacidade e segurança que emerge quando  o \textit{gateway} é mantido por empresas. Infelizmente em 2020 a Fundação Mozilla deixou o projeto \cite{MozillaIoTDisengage}, o que fez o projeto perder tração.

% (\textbf{nota 4}: Na época em que o projeto Mozilla Web of Things perdeu tração, havia dois alunos na EACH com TCC em andamento e que pretendiam usar MozWoT).

Resultados preliminares do projeto de Iniciação Científica voluntária ``Uma avaliação sobre gateways IoT e sua integração com Web Semântica'', de Daniel Macris, em desenvolvimento na EACH, atraíram atenção para um arcabouço para desenvolvimento de objetos na Web das Coisas, um desses objetos são \textit{gateways}, chamado WotPy \cite{GARCIAMANGAS2019235}. Este arcabouço foi criado no ambiente acadêmico (em contraposição a grandes comunidades ou empresas), com a intenção declarada de implementar as recomendações W3C (em contraposição a atender um determinado nicho, como domótica), implementa vários protocolos de comunicação com dispositivos como HTTP, MQTT, CoAP, é codificado em Python, encapsulado com Docker, aparentemente, não é muito extenso, e é citada na lista de aplicações para Web das Coisas da W3C \cite{WoTDevTools}. Entretanto, há dificuldade para instalação e operação na plataforma testada por Daniel.

Gustavo Ariga desenvolveu, como projeto na disciplina ACH2157-Computação Física, o projeto Protetor Solar \cite{ProtetorSolar}, onde ganhou familiaridade com dispositivos baseados no microcontrolador ESP32 e programados em MicroPython e com comunicação baseada no protocolo HTTP, tem experiência com Docker.

Há um bom ajuste entre as habilidade de Gustavo Ariga com as necessidades para criar uma nova versão de WotPy, resolvendo as dificuldades levantadas por Daniel. WotPy contém um conjunto de testes, então o indicador de sucesso é passar nos testes. Acredito que isso seja feito em um ou dois meses. No restante do semestre, gostaria que Gustavo ajustasse a comunicação do dispositivo Protetor Solar (ou similar) para que a comunicação (handshake de publicação de capacidades e transmissão de observações) fosse feita dentro da recomendação W3C-WoT. 

Supondo que este TCC atinja completo sucesso, no semestre seguinte (2023-2), Gustavo explorará a integração de WotPy com grafos de conhecimento em uma área denominada \textit{Semantic Web of Things - SWoT} \cite{Scioscia2009} \cite{Jara2014SWoT}. Os conceitos em que se baseia essa área e a justificativa para o trabalho serão detalhados no plano de atividades do próximo semestre. Nesta ocasião os resultados das atividades propostas no presente plano estarão determinados o que permitirá detalhar os passos do plano do próximo semestre com melhor qualidade. Neste momento, o que se antevê é:

\begin{itemize}
\item {Capacitação do aluno para o ferramental de grafos de conhecimento;}
\item {escolha de uma plataforma para armazenamento de grafos de conhecimento;}
\item {armazenamento das capacidades do sensor codificadas em SSN e das observações em SOSA \cite{JANOWICZ20191}, \cite{Janowicz:17:SSN}}
\item{publicização desses grafos através de um endpoint SPARQL;}
\item{definição e implementação de casos de teste da integração.}
\end{itemize}

Neste momento, acredita-se que as bibliotecas RDFLib (https://rdflib.readthedocs.io/en/stable/), OWLReady (https://owlready2.readthedocs.io/en/latest/), Flask (https://flask.palletsprojects.com/en/2.2.x/), Django (https://www.djangoproject.com/), SPARQLWrapper (https://sparqlwrapper.readthedocs.io/en/latest/) e o projeto rdflib-endpoint (https://github.com/vemonet/rdflib-endpoint) sejam subsídios para as atividades do segundo semestre.


\chapter{Objetivos}

\section{Objetivo Geral}

% (Descrição do objetivo geral do trabalho.)

O objetivo geral deste trabalho é contribuir para o desenvolvimento do WoTPy \url{https://github.com/agmangas/wot-py}, um \textit{gateway} experimental baseado no W3C-WoT, que melhora a interoperabilidade entre dispositivos IoT de diferentes fabricantes, facilitando a comunicação e integração em uma única aplicação ou sistema.

\section{Objetivo Específico}

% (Descrição dos objetivos específicos do trabalho.)

Para atingir o objetivo geral, os seguintes objetivos específicos foram estabelecidos:

\begin{enumerate}
    \item Analisar as especificações do W3C-WoT \cite{WoTArchitecture} e compreender os principais conceitos e requisitos para a implementação do WoTPy;
    \item Selecionar e estudar as bibliotecas e ferramentas das dependências do WoTPy;
    \item Resolver problemas de instalação, garantindo que o WoTPy possa ser facilmente implantado e configurado em diferentes ambientes e sistemas;
    \item Testar e validar as contribuições feitas;
    \item Desenvolver exemplo de uso e documentação detalhada para auxiliar desenvolvedores e usuários na implementação do WoTPy em seus projetos IoT;
\end{enumerate}

Caso esses objetivos específicos sejam cumpridos no primeiro semestre de 2023, o objetivo para o próximo semestre (segundo semestre de 2023) é abordar a integração do WoTPy com grafos de conhecimento. Essa integração visa melhorar ainda mais a interoperabilidade dos dispositivos IoT e possibilitar a divulgação das capacidades dos sensores e suas observações através de um endpoint SPARQL.


\chapter{Relevância ou Justificativa}

% (Por que é importante realizar este trabalho? Quais são os benefícios e quem pode se beneficiar dos resultados deste projeto?)

A realização deste trabalho é de grande relevância devido à crescente demanda por soluções que facilitem a integração e interoperabilidade entre dispositivos IoT de diferentes fabricantes. À medida que a Internet das Coisas se expande e se torna cada vez mais comum, é fundamental abordar as dificuldades e limitações atuais de comunicação e integração entre dispositivos IoT como citado em \cite{Stirbu2008} \cite{Gyrard2017}. \cite{GARCIAMANGAS2019235}, \cite{OpenApíWoT2021}.

Ao fazer a contribuição para WoTPy, é possível oferecer um potencial de solução arquitetônica eficiente e padronizada para conectar dispositivos IoT, independentemente do fabricante, o que facilita a criação de aplicações e sistemas unificados. Essa padronização promovida pela Web das Coisas (WoT) também contribui para a democratização do acesso à tecnologia IoT, permitindo que desenvolvedores e usuários finais possam explorar e implementar soluções IoT de maneira mais eficiente e acessível.

Os benefícios esperados deste trabalho incluem:

\begin{itemize}
    \item Facilitar a instalação do WoTPy, tornando-o mais acessível para desenvolvedores e usuários finais;
    \item Fornecer documentação detalhada e exemplo prático para apoiar a adoção do WoTPy em projetos IoT.
\end{itemize}

Os possíveis beneficiários dos resultados deste projeto abrangem uma ampla gama de setores e atores, incluindo:

\begin{itemize}
    \item Fabricantes de dispositivos IoT que buscam simplificar a integração de seus produtos com outros dispositivos e sistemas;
    \item Desenvolvedores de software e engenheiros de sistemas interessados em construir soluções IoT eficientes e interoperáveis;
    \item Empresas e organizações que buscam implementar soluções IoT em seus processos e infraestruturas;
    \item Pesquisadores e acadêmicos que estudam e desenvolvem novas abordagens para a Internet das Coisas e a Web das Coisas.
\end{itemize}

A integração do WoTPy com grafos de conhecimento tem como objetivo melhorar ainda mais a interoperabilidade entre dispositivos IoT de diferentes fabricantes, permitindo uma comunicação e integração mais eficiente e semântica entre eles.

A área da SWOT \cite{Scioscia2009} \cite{Jara2014SWoT} é baseada em conceitos que visam aproveitar o potencial da Web Semântica para melhorar a interação entre dispositivos IoT. Ao adotar os princípios e tecnologias da Web Semântica, é possível criar uma camada de conhecimento compartilhado e representações semânticas das informações dos dispositivos IoT \cite{bernerslee2001semantic}. Essa abordagem facilita a busca, integração e análise de dados, permitindo a criação de aplicações e serviços mais inteligentes e personalizados.


\include{sections/métodos}
\chapter{Resultados Esperados}

% (Apresentação dos resultados esperados decorrentes do trabalho a ser desenvolvido.)

Os resultados esperados decorrentes do trabalho a ser desenvolvido abrangem várias áreas relacionadas à implementação e melhoria do WoTPy como um gateway experimental baseado no W3C-WoT. Estes resultados estão alinhados aos objetivos específicos do trabalho e visam impactar positivamente a interoperabilidade e integração de dispositivos IoT de diferentes fabricantes. A seguir, são apresentados os principais resultados esperados:

\begin{enumerate}
    \item Compreensão detalhada das especificações do W3C-WoT, possibilitando a implementação correta e eficiente do WoTPy de acordo com os padrões e diretrizes estabelecidos pelo W3C;
    \item Seleção e domínio das bibliotecas e ferramentas adequadas para o desenvolvimento do WoTPy, garantindo o suporte aos protocolos de comunicação necessários e a facilidade de integração com outros projetos IoT;
    \item Resolução dos problemas de instalação do WoTPy, tornando-o facilmente implantável e configurável em diferentes ambientes e sistemas operacionais, o que facilita a adoção do projeto por desenvolvedores e usuários finais;
    \item Desenvolvimento de exemplos de uso e documentação detalhada para auxiliar outros desenvolvedores e usuários na implementação do WoTPy em seus próprios projetos IoT, promovendo a disseminação e adoção do projeto pela comunidade;
    \item Validação das contribuições realizadas por meio do conjunto de testes do WoTPy.
\end{enumerate}

Com a obtenção desses resultados, espera-se que o WoTPy se torne um gateway funcional e eficaz, capaz de facilitar a comunicação e integração entre dispositivos IoT de diferentes fabricantes. Além disso, espera-se que o trabalho contribua para a popularização do WoTPy e do conceito de Web das Coisas, melhorando a interoperabilidade e a eficiência dos sistemas IoT em geral.

 
\chapter{Cronograma}

%
% https://www.overleaf.com/latex/examples/gantt-charts-with-the-pgfgantt-package/jmkwfxrnfxnw
% A fairly complicated example from section 2.9 of the package
% documentation. This reproduces an example from Wikipedia:
% http://en.wikipedia.org/wiki/Gantt_chart
%

\definecolor{barblue}{RGB}{153,204,254}
\definecolor{groupblue}{RGB}{51,102,254}
\definecolor{linkred}{RGB}{165,0,33}
\renewcommand\sfdefault{phv}
\renewcommand\mddefault{mc}
\renewcommand\bfdefault{bc}
\setganttlinklabel{s-s}{START-TO-START}
\setganttlinklabel{f-s}{FINISH-TO-START}
\setganttlinklabel{f-f}{FINISH-TO-FINISH}
\sffamily
\begin{adjustbox}{width=1.1\textwidth,center}
\begin{ganttchart}[
    canvas/.append style={fill=none, draw=black!5, line width=.75pt},
    hgrid style/.style={draw=black!5, line width=.75pt},
    vgrid={*1{draw=black!5, line width=.75pt}},
    today=4,
    today rule/.style={
      draw=black!64,
      dash pattern=on 3.5pt off 4.5pt,
      line width=1.5pt
    },
    today label font=\small\bfseries,
    title/.style={draw=none, fill=none},
    title label font=\bfseries\footnotesize,
    title label node/.append style={below=7pt},
    include title in canvas=false,
    bar label font=\mdseries\small\color{black!70},
    bar label node/.append style={left=2cm},
    bar/.append style={draw=none, fill=black!63},
    bar incomplete/.append style={fill=barblue},
    bar progress label font=\mdseries\footnotesize\color{black!70},
    group incomplete/.append style={fill=groupblue},
    group left shift=0,
    group right shift=0,
    group height=.5,
    group peaks tip position=0,
    group label node/.append style={left=.6cm},
    group progress label font=\bfseries\small,
    link/.style={-latex, line width=1.5pt, linkred},
    link label font=\scriptsize\bfseries,
    link label node/.append style={below left=-2pt and 0pt}
  ]{1}{12}
  \gantttitle[
    title label node/.append style={below left=7pt and -3pt}
  ]{MESES:\quad1}{1}
  \gantttitlelist{2,...,12}{1} \\
  
  \ganttgroup[progress=10]{WBS 1 Primeiro Semestre}{4}{7} \\
  \ganttbar[
    progress=30,
    name=WBS1A
  ]{\textbf{WBS 1.1} Analisar Especificações do W3C-WoT}{4}{4} \\
  \ganttbar[
    progress=40,
    name=WBS1B
  ]{\textbf{WBS 1.2} Analisar Dependências do WoTPy}{4}{4} \\
  \ganttbar[
    progress=0,
    name=WBS1C
  ]{\textbf{WBS 1.3} Resolver Problemas de Instalação}{5}{6} \\
  \ganttbar[
    progress=0,
    name=WBS1D
  ]{\textbf{WBS 1.4} Realizar Testes de Validação}{5}{6} \\
  \ganttbar[
    progress=0,
    name=WBS1E
  ]{\textbf{WBS 1.5} Desenvolver Exemplos de Uso}{6}{7} \\
  
  \ganttgroup[progress=0]{WBS 2 Segundo Semestre}{8}{12} \\
    \ganttbar[
    progress=0,
    name=WBS2A
  ]{\textbf{WBS 2.1} Capacitar em Grafos de Conhecimento}{8}{8} \\
  \ganttbar[
    progress=0,
    name=WBS2B
  ]{\textbf{WBS 2.2} Escolher Plataforma}{9}{9} \\
  \ganttbar[
    progress=0,
    name=WBS2C
  ]{\textbf{WBS 2.3} Armazenar Capacidades do Sensor e Observações}{9}{10} \\
  \ganttbar[
    progress=0,
    name=WBS2D
  ]{\textbf{WBS 2.4} Publicizar Grafos}{10}{12} \\
  \ganttbar[
    progress=0,
    name=WBS2E
  ]{\textbf{WBS 2.5} Definir e Implementar Casos de Teste}{12}{12} \\

  \ganttlink[link type=s-s]{WBS1A}{WBS1B}
  \ganttlink[link type=f-s]{WBS1B}{WBS1C}
  \ganttlink[link type=f-f]{WBS1C}{WBS1D}
  \ganttlink[link type=f-s]{WBS1D}{WBS1E}

  \ganttlink[link type=f-s]{WBS2A}{WBS2B}
  \ganttlink[link type=f-s]{WBS2B}{WBS2C}
  \ganttlink[link type=f-s]{WBS2C}{WBS2D}
  \ganttlink[link type=f-s]{WBS2D}{WBS2E}
  
  % \ganttlink[
    % link type=f-f,
    % link label node/.append style=left
  % {WBS1C}{WBS1D}
  
\end{ganttchart}
\end{adjustbox}

%
% A simpler example from the package documentation:
%

% \begin{ganttchart}{1}{12}
  % \gantttitle{2011}{12} \\
  % \gantttitlelist{1,...,12}{1} \\
  % \ganttgroup{Group 1}{1}{7} \\
  % \ganttbar{Task 1}{1}{2} \\
  % \ganttlinkedbar{Task 2}{3}{7} \ganttnewline
  % \ganttmilestone{Milestone}{7} \ganttnewline
  % \ganttbar{Final Task}{8}{12}
  % \ganttlink{elem2}{elem3}
  % \ganttlink{elem3}{elem4}
% \end{ganttchart}

% \end{document}

No início do primeiro semestre, uma análise detalhada das especificações do W3C-WoT será realizada (\ref{estudo}), seguida pela seleção e estudo das bibliotecas e ferramentas relevantes para o projeto (\ref{dependência}) Em paralelo, serão identificados e resolvidos problemas de instalação do WoTPy (\ref{instalação}). Com as melhorias implementadas, serão realizados testes e validações para garantir a qualidade e a funcionalidade das contribuições feitas (\ref{teste}).

No final do primeiro semestre, serão desenvolvidos exemplos de uso e documentação detalhada do projeto (\ref{uso}), auxiliando desenvolvedores e usuários na implementação do WoTPy em seus próprios projetos IoT. O exemplo prático incluirá ajustes na comunicação do dispositivo Protetor Solar (ou similar).

Para o segundo semestre, o que se espera inclui a capacitação em ferramentas de grafos de conhecimento, a escolha de uma plataforma de armazenamento para esses grafos, o armazenamento das capacidades do sensor em SSN e das observações em SOSA, a publicização desses grafos por meio de um endpoint SPARQL e a definição e implementação de casos de teste para a integração.




% ----------------------------------------------------------
% ELEMENTOS PÓS-TEXTUAIS
% ----------------------------------------------------------
\postextual
% ----------------------------------------------------------

% ----------------------------------------------------------
% Referências bibliográficas
% ----------------------------------------------------------
\bibliography{ref}


%---------------------------------------------------------------------
% INDICE REMISSIVO
%---------------------------------------------------------------------
\phantompart
\printindex
%---------------------------------------------------------------------

\end{document}

