\chapter{Resultados Esperados}

% (Apresentação dos resultados esperados decorrentes do trabalho a ser desenvolvido.)

Os resultados esperados decorrentes do trabalho a ser desenvolvido abrangem várias áreas relacionadas à implementação e melhoria do WoTPy como um gateway experimental baseado no W3C-WoT. Estes resultados estão alinhados aos objetivos específicos do trabalho e visam impactar positivamente a interoperabilidade e integração de dispositivos IoT de diferentes fabricantes. A seguir, são apresentados os principais resultados esperados:

\begin{enumerate}
    \item Compreensão detalhada das especificações do W3C-WoT, possibilitando a implementação correta e eficiente do WoTPy de acordo com os padrões e diretrizes estabelecidos pelo W3C;
    \item Seleção e domínio das bibliotecas e ferramentas adequadas para o desenvolvimento do WoTPy, garantindo o suporte aos protocolos de comunicação necessários e a facilidade de integração com outros projetos IoT;
    \item Resolução dos problemas de instalação do WoTPy, tornando-o facilmente implantável e configurável em diferentes ambientes e sistemas operacionais, o que facilita a adoção do projeto por desenvolvedores e usuários finais;
    \item Desenvolvimento de exemplos de uso e documentação detalhada para auxiliar outros desenvolvedores e usuários na implementação do WoTPy em seus próprios projetos IoT, promovendo a disseminação e adoção do projeto pela comunidade;
    \item Validação das contribuições realizadas por meio do conjunto de testes do WoTPy.
\end{enumerate}

Com a obtenção desses resultados, espera-se que o WoTPy se torne um gateway funcional e eficaz, capaz de facilitar a comunicação e integração entre dispositivos IoT de diferentes fabricantes. Além disso, espera-se que o trabalho contribua para a popularização do WoTPy e do conceito de Web das Coisas, melhorando a interoperabilidade e a eficiência dos sistemas IoT em geral.

 