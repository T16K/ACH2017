\chapter{Resumo}

% (Resumo do trabalho.)

% Introdução
À medida que a Internet das Coisas (IoT) se populariza, cresce também a demanda por sistemas capazes de integrar dispositivos de várias marcas em uma aplicação unificada. No entanto, as restrições atuais da IoT tornam difícil a interoperabilidade entre esses dispositivos. Para contornar esse problema, o consórcio World Wide Web (W3C) \cite{W3CAbout} sugeriu a Web das Coisas (WoT) \cite{WoTCommunityWiki} como uma solução arquitetônica que utiliza a Web para promover a comunicação eficiente entre dispositivos IoT.

% Objetivos
Dentro desse cenário, o presente trabalho tem como objetivo implantar e aprimorar o WoTPy \cite{GARCIAMANGAS2019235}, um gateway experimental baseado no W3C-WoT \cite{WoTArchitecture}, compatível com protocolos como HTTP, Websockets, MQTT e CoAP. O plano de atividades abrange a solução dos problemas de instalação do WoTPy \url{https://github.com/agmangas/wot-py}, a documentação do processo e a elaboração de exemplo prático de uso. Adicionalmente, caso tudo ocorra como planejado, no próximo semestre (segundo semestre de 2023), será abordada a integração do WoTPy com grafos de conhecimento, visando melhorar a interoperabilidade dos dispositivos IoT e possibilitar a divulgação das capacidades dos sensores e suas observações através de um endpoint SPARQL \url{https://sparqlwrapper.readthedocs.io/en/latest/}.

% Conclusão
Ao longo do projeto, serão investigados os materiais e métodos necessários para alcançar os objetivos estabelecidos, como a seleção de bibliotecas adequadas e das dependências. O resultado esperado é um gateway WoTPy funcional, que possa ser facilmente instalado e utilizado para aprimorar a integração e comunicação entre dispositivos IoT de diferentes fabricantes.

