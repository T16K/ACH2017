% ------------------------------------------------------------------------
% ------------------------------------------------------------------------
% abnTeX2: Modelo de Trabalho Academico (tese de doutorado, dissertacao de
% mestrado e trabalhos monograficos em geral) em conformidade com 
% ABNT NBR 14724:2011: Informacao e documentacao - Trabalhos academicos -
% Apresentacao
%
% baseado no modelo "Template para elaboração de trabalho acadêmico - versão setembro/2016" fornecido pela Biblioteca do IFSC - http://www.ifsc.edu.br/menu-colecao-abnt (acessado em 2017-08-23)
% ------------------------------------------------------------------------
% ------------------------------------------------------------------------
\documentclass[
	% -- opções da classe memoir --
	10pt,				% tamanho da fonte
	oneside,			% para impressão só frente
	a4paper,			% tamanho do papel. 
	% -- opções da classe abntex2 --
	chapter=TITLE,		% títulos de capítulos convertidos em letras maiúsculas
	%section=TITLE,		% títulos de seções convertidos em letras maiúsculas
	%subsection=TITLE,	% títulos de subseções convertidos em letras maiúsculas
	%subsubsection=TITLE,% títulos de subsubseções convertidos em letras maiúsculas
	% -- opções do pacote babel --
	english,			% idioma adicional para hifenização
	brazil				% o último idioma é o principal do documento
	]{abntex2}

%	Todas as indicações de pacotes e configurações estão no arquivo de estilo
%  chamado estilo-monografia-ifsp.sty.
\usepackage{preamble}	

% \documentclass[tikz]{standalone}
\usepackage{pgfgantt} % https://pt.overleaf.com/latex/examples/gantt-charts-with-the-pgfgantt-package/jmkwfxrnfxnw
% \title{Gantt Charts with the pgfgantt Package}
% \begin{document}

\usepackage{xcolor} % text colors https://www.overleaf.com/learn/latex/Using_colours_in_LaTeX

%---------------------------------------------------------------------%
%---------------------------------------------------------------------%
% Informações de dados para CAPA e FOLHA DE ROSTO
%---------------------------------------------------------------------%
%---------------------------------------------------------------------%

\titulo{Integração do WoTPy com Grafos de Conhecimento}
\autor{Gustavo Tsuyoshi Ariga}
\local{São Paulo, SP}
\data{2023}
\orientador{Professores Fábio Nakano e José de Jesús Pérez-Alcázar}
\instituicao{%
  Universidade de São Paulo - USP
  \par
  Escola de Artes, Ciências e Humanidades
  \par
  Graduação em Sistemas de Informação}
\tipotrabalho{Plano de Atividades}

% O preambulo deve conter o tipo do trabalho, o objetivo, 
% o nome da instituição e a área de concentração 
\preambulo{Plano de atividades apresentado como parte dos requisitos necessários para cumprimento da disciplina ACH2017 ou ACH2018 - Projeto Supervisionado ou de Graduação I ou II.}
%---------------------------------------------------------------------%
%\textoaprovacao{}


%---------------------------------------------------------------------%
% Início do documento
%---------------------------------------------------------------------%


\begin{document}
% Seleciona o idioma do documento (conforme pacotes do babel)
\selectlanguage{brazil}
% Retira espaço extra obsoleto entre as frases.
\frenchspacing 


% ----------------------------------------------------------
% ELEMENTOS PRÉ-TEXTUAIS
% ----------------------------------------------------------
% \pretextual

\imprimircapa
% Folha de rosto - (o * indica que haverá a ficha bibliográfica)
\imprimirfolhaderosto*
% ---
%---------------------------------------------------------------------%


% ----------------------------------------------------------
% Inclusão dos capítulos que estão em outros arquivos .tex
% ----------------------------------------------------------
\chapter{Resumo}

% (Resumo do trabalho.)

A acelerada expansão da Internet das Coisas (IoT) tem impulsionado a demanda por plataformas e ferramentas que garantam uma comunicação integrada e transparente entre dispositivos oriundos de distintos fabricantes. O WoTPy \citeonline{GARCIAMANGAS2019235}, um \textit{gateway} experimental fundamentado no padrão W3C-WoT \citeonline{WoTArchitecture}, manifesta-se como uma resposta saliente a essa necessidade. No primeiro semestre de 2023, o projeto concentrou-se meticulosamente na internalização e implementação das bases fundamentais do WoTPy, abordando as rigorosas especificações do W3C-WoT, a criteriosa seleção e análise de bibliotecas correlatas, a solução de obstáculos associados à instalação e a meticulosa validação das contribuições realizadas.

Para o segundo semestre de 2023, propõe-se uma ampliação robusta das funcionalidades intrínsecas ao WoTPy, direcionando esforços para sua integrada relação com grafos de conhecimento. Tais grafos, estruturas de dados consagradas por modelar, representar e inquirir informações de natureza complexa, detêm o potencial de elevar significativamente a capacidade de interoperabilidade na esfera da IoT. Através da amalgamação do WoTPy com esses grafos, aspira-se engendrar um sistema em que os dispositivos, além de estabelecerem comunicação, possam contextualizar e interpretar dados com uma sofisticação sem precedentes.

Salienta-se, como ponto nevrálgico desta etapa, a implementação de um endpoint SPARQL \citeonline{sparqlwrapper} – linguagem de consulta estandardizada para bases de dados que empregam grafos de conhecimento. Esta implementação propiciará que usuários e profissionais do setor realizem consultas acerca das capacidades dos sensores, suas respectivas observações e as inter-relações entre distintos dispositivos de maneira padronizada e intuitiva.

Esta sofisticada integração visa catapultar o WoTPy ao patamar de ferramenta essencial à comunidade IoT, transcendendo a mera comunicação entre dispositivos. Enfatiza-se a busca por uma compreensão mais profunda e contextualizada dos dados, facilitando assim o desenvolvimento de aplicações IoT dotadas de maior inteligência, adaptabilidade e relevância contextual. Consequentemente, o projeto para o segundo semestre aspira consolidar o WoTPy não apenas como uma ferramenta de integração, mas como um instrumento primordial para a decodificação e acessibilidade do vasto universo da IoT para pesquisadores, desenvolvedores e usuários finais.


\chapter{Palavras Chaves}

% (Lista de palavras-chaves.)

Internet das Coisas (IoT), Web das Coisas (WoT), World Wide Web Consortium (W3C), WoTPy, Grafos de Conhecimento, Endpoint SPARQL.


\chapter{Modalidade}

\begin{description}
	\item (  ) Trabalho de Graduação Curto – 1 semestre - individual
	\item (  ) Trabalho de Graduação Longo (parte 1) – 1 ano – individual
	\item (x) Trabalho de Graduação Longo (parte 2) – 1 ano - individual			
	\item (  ) Trabalho de Graduação Curto – 1 semestre – grupo
	\item (  ) Trabalho de Graduação Longo (parte 1) – 1 ano – grupo
	\item (  ) Trabalho de Graduação Longo (parte 2) – 1 ano - grupo
\end{description}


\chapter{Apresentação do Problema}

A "Web das Coisas" é um termo que representa a integração de sensores, atuadores, objetos físicos e seres humanos usando tecnologias web  \cite{WoTTerminology}, \cite{WoTCommunityWiki}. O W3C define e promove padrões para a Web das Coisas, enfatizando interoperabilidade e o uso de tecnologias populares como HTTP e JavaScript.

A principal questão é como conectar diferentes redes e dispositivos. O W3C, com contribuições de empresas e pesquisadores, trabalha na criação de padrões para a Web das Coisas, visando uma abordagem amplamente aceita e duradoura \cite{Stirbu2008} \cite{Gyrard2017}. \cite{GARCIAMANGAS2019235}, \cite{OpenApíWoT2021}.

É evidente a necessidade de dispositivos que façam a ponte entre redes internas e a Internet, comumente referidos como "gateways". Esses dispositivos desempenham um papel crucial na interoperabilidade, segurança de dados e interação com o usuário \cite{Stirbu2008}, \cite{Gyrard2017}, \cite{GARCIAMANGAS2019235}.

A Semantic Web of Things (SWoT) representa uma inovação notável, situada na confluência da Internet das Coisas (IoT) e da Web Semântica \cite{ssw1}. A plena compreensão do impacto e dos princípios fundamentais dessa combinação exige uma exploração separada de suas componentes subjacentes: IoT e Web Semântica.

A IoT refere-se à integração de dispositivos e objetos do dia a dia à rede mundial, conferindo-lhes inteligência e conectividade \cite{ssw2}. Esta revolução tecnológica permite que tais dispositivos coletem, transmitam e atuem com base em dados do ambiente circundante, tudo sem intervenção humana direta \cite{ssw3}. No entanto, a diversidade de dispositivos, padrões e fabricantes na IoT introduz complexidades, principalmente quanto à interoperabilidade.

Por outro lado, a Web Semântica, uma evolução da web atual, ambiciona tornar as informações online acessíveis não apenas a humanos, mas também a máquinas \cite{ssw4}. Por meio de padrões como RDF e OWL, esta tecnologia descreve conceitos, relações e significados \cite{ssw5}. A visão é que os dados na web sejam complementados com metainformações semânticas, permitindo um processamento e compreensão mais profundos por parte das máquinas.

É nesse cenário que o SWoT ganha relevância, propondo-se a superar desafios de interoperabilidade na IoT, utilizando os princípios e técnicas da Web Semântica \cite{ssw6}. Integrando semântica aos dispositivos da IoT, passamos a ter uma descrição mais precisa não apenas dos dados gerados, mas também dos serviços ofertados, capacidades e relações entre dispositivos, tudo de forma padronizada e máquina-legível.


\chapter{Objetivos}

O presente relatório delineia metas meticulosamente articuladas, destinadas a serem atingidas ao longo do segundo semestre de 2023. Com a intenção de promover uma evolução significativa no WoTPy, o conjunto de objetivos apresentados busca atender às necessidades emergentes e multifacetadas da comunidade IoT. A seguir, detalham-se o objetivo geral e os objetivos específicos propostos:

\section{Objetivo Geral}

Refinar e expandir as capacidades do WoTPy, viabilizando sua integração avançada com grafos de conhecimento, visando posicioná-lo como uma ferramenta insuperável, capaz de responder de forma mais sofisticada às demandas da comunidade IoT.

\section{Objetivo Espcífico}

\begin{itemize}
    \item Exploração Detalhada de Grafos de Conhecimento \label{estudo}
    \begin{itemize}
        \item Profundar a investigação acadêmica sobre grafos de conhecimento, com ênfase em suas potenciais aplicações no âmbito da IoT.
        \item Adquirir proficiência nas tecnologias emergentes e padrões estabelecidos relacionados a grafos de conhecimento e SPARQL.
    \end{itemize}
    \item Integração entre o Grafo de Conhecimento e o WoTPy \label{integração}
    \begin{itemize}
        \item Concretizar a fusão do WoTPy com sistemas de grafos de conhecimento, estabelecendo um mecanismo robusto que facilite consultas e operações avançadas.
    \end{itemize}
    \item Desenvolvimento do Endpoint SPARQL \label{endpoint}
    \begin{itemize}
        \item Adaptar o WoTPy para acolher e processar eficientemente consultas SPARQL, otimizando o acesso, manipulação e interpretação dos dados armazenados.
        \item Certificar que a plataforma possa recuperar, de forma ágil e precisa, informações pertinentes sobre sensores e respectivas observações.
    \end{itemize}
    \item Testes e Validação \label{teste}
    \begin{itemize}
        \item Submeter as inovações implementadas a testes em cenários reais, assegurando sua funcionalidade, resiliência e escalabilidade.
    \end{itemize}
    \item Documentação \label{documentação}
    \begin{itemize}
        \item Providenciar documentação e exemplo prático, fortalecendo a capacidade de desenvolvedores e usuários em explorar plenamente as novas funcionalidades.
    \end{itemize}
\end{itemize}


\chapter{Relevância ou Justificativa}

A revolução digital atual evidencia uma transição notável de um mundo conectado para um universo hiperconectado, impulsionado pela proliferação da Internet das Coisas (IoT). Neste contexto, a capacidade de garantir que dispositivos e sistemas heterogêneos conversem entre si torna-se uma necessidade imperativa, e não apenas um luxo.

A demanda emergente por uma integração eficiente, conforme indicada por \citeonline{Stirbu2008} e \citeonline{Gyrard2017}, coloca em destaque o desafio da interoperabilidade. Isso é ainda mais premente à luz dos desafios identificados em estudos como \citeonline{GARCIAMANGAS2019235} e as diretrizes propostas por \cite{OpenApíWoT2021}.

No âmago deste trabalho, o WoTPy surge como um catalisador para uma integração mais fluida no domínio da IoT. Oferecendo uma estrutura arquitetônica coesa e padronizada, a proposta transcende os constrangimentos tradicionais associados à diversidade de fabricantes, levando a uma visão unificada e harmonizada para aplicações e sistemas IoT. Em essência, isso não apenas facilita a implementação técnica, mas também amplifica o impacto social da IoT, democratizando seu acesso e uso.

Este trabalho, portanto, traz à tona benefícios tangíveis:

\begin{itemize}
\item Uma ampliação da usabilidade do WoTPy, tornando-o uma solução mais abrangente e amigável para diferentes públicos;
\item A disposição de recursos didáticos, como documentação e exemplos práticos, para aprimorar o entendimento e a adoção do WoTPy.
\end{itemize}

Ao refletirmos sobre os potenciais beneficiários deste projeto, percebemos a universalidade de sua aplicação. Desde fabricantes de dispositivos IoT que aspiram por uma integração mais simplificada, passando por desenvolvedores e engenheiros dedicados à inovação, até organizações empresariais e instituições acadêmicas, a gama de partes interessadas é vasta e diversificada.


\include{sections/07_métodos}
\chapter{Resultados Esperados}

O projeto visa integrar WoTPy com grafos de conhecimento, buscando promover uma melhor interoperabilidade entre dispositivos IoT de diferentes fabricantes e potencializar a eficácia na comunicação e integração destes dispositivos. Neste contexto, os resultados esperados são multifacetados, abrangendo aspectos técnicos e aplicativos.

\section{Contribuições Técnicas}

\begin{itemize}
    \item \textbf{Integração Efetiva:} A conclusão bem-sucedida da integração do WoTPy com sistemas de grafos de conhecimento, resultando em uma solução robusta e estável.
    \item \textbf{Desenvolvimento de Endpoint SPARQL:} Implementação de um endpoint SPARQL eficiente, permitindo consultas dinâmicas sobre capacidades de sensores e suas observações.
    \item \textbf{Documentação Detalhada:} Disponibilização de documentação técnica abrangente, garantindo que os desenvolvedores possam entender, adotar e estender a solução proposta.
\end{itemize}

\section{Aplicações e Usabilidade}

\begin{itemize}
    \item \textbf{Interoperabilidade Aprimorada:} Dispositivos IoT de diferentes fabricantes poderão se comunicar e integrar de maneira mais eficiente, impulsionando a criação de sistemas IoT unificados.
    \item \textbf{Aplicações Semânticas:} A capacidade de explorar os benefícios da Web Semântica na IoT, possibilitando o desenvolvimento de aplicações mais inteligentes, personalizadas e contextuais.
    \item \textbf{Exemplo Prático:} Disponibilização de exemplo prático que demonstra o potencial e as capacidades da solução, facilitando sua adoção em projetos reais.
\end{itemize}


\chapter{Cronograma}

%
% https://www.overleaf.com/latex/examples/gantt-charts-with-the-pgfgantt-package/jmkwfxrnfxnw
% A fairly complicated example from section 2.9 of the package
% documentation. This reproduces an example from Wikipedia:
% http://en.wikipedia.org/wiki/Gantt_chart
%

\definecolor{barblue}{RGB}{153,204,254}
\definecolor{groupblue}{RGB}{51,102,254}
\definecolor{linkred}{RGB}{165,0,33}
\renewcommand\sfdefault{phv}
\renewcommand\mddefault{mc}
\renewcommand\bfdefault{bc}
\setganttlinklabel{s-s}{START-TO-START}
\setganttlinklabel{f-s}{FINISH-TO-START}
\setganttlinklabel{f-f}{FINISH-TO-FINISH}
\sffamily
\begin{adjustbox}{width=1.1\textwidth,center}
\begin{ganttchart}[
    canvas/.append style={fill=none, draw=black!5, line width=.75pt},
    hgrid style/.style={draw=black!5, line width=.75pt},
    vgrid={*1{draw=black!5, line width=.75pt}},
    today=4,
    today rule/.style={
      draw=black!64,
      dash pattern=on 3.5pt off 4.5pt,
      line width=1.5pt
    },
    today label font=\small\bfseries,
    title/.style={draw=none, fill=none},
    title label font=\bfseries\footnotesize,
    title label node/.append style={below=7pt},
    include title in canvas=false,
    bar label font=\mdseries\small\color{black!70},
    bar label node/.append style={left=2cm},
    bar/.append style={draw=none, fill=black!63},
    bar incomplete/.append style={fill=barblue},
    bar progress label font=\mdseries\footnotesize\color{black!70},
    group incomplete/.append style={fill=groupblue},
    group left shift=0,
    group right shift=0,
    group height=.5,
    group peaks tip position=0,
    group label node/.append style={left=.6cm},
    group progress label font=\bfseries\small,
    link/.style={-latex, line width=1.5pt, linkred},
    link label font=\scriptsize\bfseries,
    link label node/.append style={below left=-2pt and 0pt}
  ]{1}{12}
  \gantttitle[
    title label node/.append style={below left=7pt and -3pt}
  ]{MESES:\quad1}{1}
  \gantttitlelist{2,...,12}{1} \\
  
  \ganttgroup[progress=100]{WBS 1 Primeiro Semestre}{4}{7} \\
  \ganttbar[
    progress=100,
    name=WBS1A
  ]{\textbf{WBS 1.1} Analisar Especificações do W3C-WoT}{4}{4} \\
  \ganttbar[
    progress=100,
    name=WBS1B
  ]{\textbf{WBS 1.2} Analisar Dependências do WoTPy}{4}{4} \\
  \ganttbar[
    progress=100,
    name=WBS1C
  ]{\textbf{WBS 1.3} Resolver Problemas de Instalação}{5}{6} \\
  \ganttbar[
    progress=100,
    name=WBS1D
  ]{\textbf{WBS 1.4} Realizar Testes de Validação}{5}{6} \\
  \ganttbar[
    progress=100,
    name=WBS1E
  ]{\textbf{WBS 1.5} Desenvolver Exemplos de Uso}{6}{7} \\
  
  \ganttgroup[progress=40]{WBS 2 Segundo Semestre}{8}{12} \\
    \ganttbar[
    progress=100,
    name=WBS2A
  ]{\textbf{WBS 2.1} Exploração Detalhada de Grafos de Conhecimento}{8}{8} \\
  \ganttbar[
    progress=100,
    name=WBS2B
  ]{\textbf{WBS 2.2} Integração entre o Grafo de Conhecimento e o WoTPy}{9}{9} \\
  \ganttbar[
    progress=40,
    name=WBS2C
  ]{\textbf{WBS 2.3} Desenvolvimento do Endpoint SPARQL}{9}{10} \\
  \ganttbar[
    progress=0,
    name=WBS2D
  ]{\textbf{WBS 2.4} Teste e Validação}{10}{12} \\
  \ganttbar[
    progress=0,
    name=WBS2E
  ]{\textbf{WBS 2.5} Documentação}{12}{12} \\

  \ganttlink[link type=s-s]{WBS1A}{WBS1B}
  \ganttlink[link type=f-s]{WBS1B}{WBS1C}
  \ganttlink[link type=f-f]{WBS1C}{WBS1D}
  \ganttlink[link type=f-s]{WBS1D}{WBS1E}

  \ganttlink[link type=f-s]{WBS2A}{WBS2B}
  \ganttlink[link type=f-s]{WBS2B}{WBS2C}
  \ganttlink[link type=f-s]{WBS2C}{WBS2D}
  \ganttlink[link type=f-s]{WBS2D}{WBS2E}
  
  % \ganttlink[
    % link type=f-f,
    % link label node/.append style=left
  % {WBS1C}{WBS1D}
  
\end{ganttchart}
\end{adjustbox}

%
% A simpler example from the package documentation:
%

% \begin{ganttchart}{1}{12}
  % \gantttitle{2011}{12} \\
  % \gantttitlelist{1,...,12}{1} \\
  % \ganttgroup{Group 1}{1}{7} \\
  % \ganttbar{Task 1}{1}{2} \\
  % \ganttlinkedbar{Task 2}{3}{7} \ganttnewline
  % \ganttmilestone{Milestone}{7} \ganttnewline
  % \ganttbar{Final Task}{8}{12}
  % \ganttlink{elem2}{elem3}
  % \ganttlink{elem3}{elem4}
% \end{ganttchart}

% \end{document}

Ao dar início ao segundo semestre, será promovida uma revisão meticulosa da literatura pertinente que envolve a Web das Coisas (WoT), WoTPy e grafos de conhecimento (\ref{estudo}). Esta revisão tem por objetivo consolidar o entendimento teórico, garantindo assim uma base sólida para as etapas subsequentes. Subsequentemente, será implementado o processo de integração do WoTPy com os grafos de conhecimento (\ref{integração}), visando uma harmonização eficaz entre estas duas ferramentas essenciais.

Posteriormente, a atenção será dirigida ao desenvolvimento do endpoint SPARQL (\ref{endpoint}), assegurando uma interface robusta e eficiente para consultas.

Ao nos aproximarmos do encerramento do segundo semestre, uma série de testes e validações será conduzida com o intuito de certificar que a solução elaborada atenda de forma precisa às necessidades previamente estabelecidas (\ref{teste}). Por fim, será redigida uma documentação detalhada e rigorosa, garantindo que os usuários e interessados possam compreender, implementar e se beneficiar da solução proposta (\ref{documentação}).




% ----------------------------------------------------------
% ELEMENTOS PÓS-TEXTUAIS
% ----------------------------------------------------------
\postextual
% ----------------------------------------------------------

% ----------------------------------------------------------
% Referências bibliográficas
% ----------------------------------------------------------
\bibliography{ref}


%---------------------------------------------------------------------
% INDICE REMISSIVO
%---------------------------------------------------------------------
\phantompart
\printindex
%---------------------------------------------------------------------

\end{document}

