\chapter{Resumo}

% (Resumo do trabalho.)

A acelerada expansão da Internet das Coisas (IoT) tem impulsionado a demanda por plataformas e ferramentas que garantam uma comunicação integrada e transparente entre dispositivos oriundos de distintos fabricantes. O WoTPy \citeonline{GARCIAMANGAS2019235}, um \textit{gateway} experimental fundamentado no padrão W3C-WoT \citeonline{WoTArchitecture}, manifesta-se como uma resposta saliente a essa necessidade. No primeiro semestre de 2023, o projeto concentrou-se meticulosamente na internalização e implementação das bases fundamentais do WoTPy, abordando as rigorosas especificações do W3C-WoT, a criteriosa seleção e análise de bibliotecas correlatas, a solução de obstáculos associados à instalação e a meticulosa validação das contribuições realizadas.

Para o segundo semestre de 2023, propõe-se uma ampliação robusta das funcionalidades intrínsecas ao WoTPy, direcionando esforços para sua integrada relação com grafos de conhecimento. Tais grafos, estruturas de dados consagradas por modelar, representar e inquirir informações de natureza complexa, detêm o potencial de elevar significativamente a capacidade de interoperabilidade na esfera da IoT. Através da amalgamação do WoTPy com esses grafos, aspira-se engendrar um sistema em que os dispositivos, além de estabelecerem comunicação, possam contextualizar e interpretar dados com uma sofisticação sem precedentes.

Salienta-se, como ponto nevrálgico desta etapa, a implementação de um endpoint SPARQL \citeonline{sparqlwrapper} – linguagem de consulta estandardizada para bases de dados que empregam grafos de conhecimento. Esta implementação propiciará que usuários e profissionais do setor realizem consultas acerca das capacidades dos sensores, suas respectivas observações e as inter-relações entre distintos dispositivos de maneira padronizada e intuitiva.

Esta sofisticada integração visa catapultar o WoTPy ao patamar de ferramenta essencial à comunidade IoT, transcendendo a mera comunicação entre dispositivos. Enfatiza-se a busca por uma compreensão mais profunda e contextualizada dos dados, facilitando assim o desenvolvimento de aplicações IoT dotadas de maior inteligência, adaptabilidade e relevância contextual. Consequentemente, o projeto para o segundo semestre aspira consolidar o WoTPy não apenas como uma ferramenta de integração, mas como um instrumento primordial para a decodificação e acessibilidade do vasto universo da IoT para pesquisadores, desenvolvedores e usuários finais.

