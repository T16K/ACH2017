\chapter{Apresentação do Problema}

A "Web das Coisas" é um termo que representa a integração de sensores, atuadores, objetos físicos e seres humanos usando tecnologias web  \cite{WoTTerminology}, \cite{WoTCommunityWiki}. O W3C define e promove padrões para a Web das Coisas, enfatizando interoperabilidade e o uso de tecnologias populares como HTTP e JavaScript.

A principal questão é como conectar diferentes redes e dispositivos. O W3C, com contribuições de empresas e pesquisadores, trabalha na criação de padrões para a Web das Coisas, visando uma abordagem amplamente aceita e duradoura \cite{Stirbu2008} \cite{Gyrard2017}. \cite{GARCIAMANGAS2019235}, \cite{OpenApíWoT2021}.

É evidente a necessidade de dispositivos que façam a ponte entre redes internas e a Internet, comumente referidos como "gateways". Esses dispositivos desempenham um papel crucial na interoperabilidade, segurança de dados e interação com o usuário \cite{Stirbu2008}, \cite{Gyrard2017}, \cite{GARCIAMANGAS2019235}.

A Semantic Web of Things (SWoT) representa uma inovação notável, situada na confluência da Internet das Coisas (IoT) e da Web Semântica \cite{ssw1}. A plena compreensão do impacto e dos princípios fundamentais dessa combinação exige uma exploração separada de suas componentes subjacentes: IoT e Web Semântica.

A IoT refere-se à integração de dispositivos e objetos do dia a dia à rede mundial, conferindo-lhes inteligência e conectividade \cite{ssw2}. Esta revolução tecnológica permite que tais dispositivos coletem, transmitam e atuem com base em dados do ambiente circundante, tudo sem intervenção humana direta \cite{ssw3}. No entanto, a diversidade de dispositivos, padrões e fabricantes na IoT introduz complexidades, principalmente quanto à interoperabilidade.

Por outro lado, a Web Semântica, uma evolução da web atual, ambiciona tornar as informações online acessíveis não apenas a humanos, mas também a máquinas \cite{ssw4}. Por meio de padrões como RDF e OWL, esta tecnologia descreve conceitos, relações e significados \cite{ssw5}. A visão é que os dados na web sejam complementados com metainformações semânticas, permitindo um processamento e compreensão mais profundos por parte das máquinas.

É nesse cenário que o SWoT ganha relevância, propondo-se a superar desafios de interoperabilidade na IoT, utilizando os princípios e técnicas da Web Semântica \cite{ssw6}. Integrando semântica aos dispositivos da IoT, passamos a ter uma descrição mais precisa não apenas dos dados gerados, mas também dos serviços ofertados, capacidades e relações entre dispositivos, tudo de forma padronizada e máquina-legível.

