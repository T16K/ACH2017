\chapter{Objetivos}

O presente relatório delineia metas meticulosamente articuladas, destinadas a serem atingidas ao longo do segundo semestre de 2023. Com a intenção de promover uma evolução significativa no WoTPy, o conjunto de objetivos apresentados busca atender às necessidades emergentes e multifacetadas da comunidade IoT. A seguir, detalham-se o objetivo geral e os objetivos específicos propostos:

\section{Objetivo Geral}

Refinar e expandir as capacidades do WoTPy, viabilizando sua integração avançada com grafos de conhecimento, visando posicioná-lo como uma ferramenta insuperável, capaz de responder de forma mais sofisticada às demandas da comunidade IoT.

\section{Objetivo Espcífico}

\begin{itemize}
    \item Exploração Detalhada de Grafos de Conhecimento \label{estudo}
    \begin{itemize}
        \item Profundar a investigação acadêmica sobre grafos de conhecimento, com ênfase em suas potenciais aplicações no âmbito da IoT.
        \item Adquirir proficiência nas tecnologias emergentes e padrões estabelecidos relacionados a grafos de conhecimento e SPARQL.
    \end{itemize}
    \item Integração entre o Grafo de Conhecimento e o WoTPy \label{integração}
    \begin{itemize}
        \item Concretizar a fusão do WoTPy com sistemas de grafos de conhecimento, estabelecendo um mecanismo robusto que facilite consultas e operações avançadas.
    \end{itemize}
    \item Desenvolvimento do Endpoint SPARQL \label{endpoint}
    \begin{itemize}
        \item Adaptar o WoTPy para acolher e processar eficientemente consultas SPARQL, otimizando o acesso, manipulação e interpretação dos dados armazenados.
        \item Certificar que a plataforma possa recuperar, de forma ágil e precisa, informações pertinentes sobre sensores e respectivas observações.
    \end{itemize}
    \item Testes e Validação \label{teste}
    \begin{itemize}
        \item Submeter as inovações implementadas a testes em cenários reais, assegurando sua funcionalidade, resiliência e escalabilidade.
    \end{itemize}
    \item Documentação \label{documentação}
    \begin{itemize}
        \item Providenciar documentação e exemplo prático, fortalecendo a capacidade de desenvolvedores e usuários em explorar plenamente as novas funcionalidades.
    \end{itemize}
\end{itemize}

