\chapter{Relevância ou Justificativa}

A revolução digital atual evidencia uma transição notável de um mundo conectado para um universo hiperconectado, impulsionado pela proliferação da Internet das Coisas (IoT). Neste contexto, a capacidade de garantir que dispositivos e sistemas heterogêneos conversem entre si torna-se uma necessidade imperativa, e não apenas um luxo.

A demanda emergente por uma integração eficiente, conforme indicada por \citeonline{Stirbu2008} e \citeonline{Gyrard2017}, coloca em destaque o desafio da interoperabilidade. Isso é ainda mais premente à luz dos desafios identificados em estudos como \citeonline{GARCIAMANGAS2019235} e as diretrizes propostas por \cite{OpenApíWoT2021}.

No âmago deste trabalho, o WoTPy surge como um catalisador para uma integração mais fluida no domínio da IoT. Oferecendo uma estrutura arquitetônica coesa e padronizada, a proposta transcende os constrangimentos tradicionais associados à diversidade de fabricantes, levando a uma visão unificada e harmonizada para aplicações e sistemas IoT. Em essência, isso não apenas facilita a implementação técnica, mas também amplifica o impacto social da IoT, democratizando seu acesso e uso.

Este trabalho, portanto, traz à tona benefícios tangíveis:

\begin{itemize}
\item Uma ampliação da usabilidade do WoTPy, tornando-o uma solução mais abrangente e amigável para diferentes públicos;
\item A disposição de recursos didáticos, como documentação e exemplos práticos, para aprimorar o entendimento e a adoção do WoTPy.
\end{itemize}

Ao refletirmos sobre os potenciais beneficiários deste projeto, percebemos a universalidade de sua aplicação. Desde fabricantes de dispositivos IoT que aspiram por uma integração mais simplificada, passando por desenvolvedores e engenheiros dedicados à inovação, até organizações empresariais e instituições acadêmicas, a gama de partes interessadas é vasta e diversificada.

