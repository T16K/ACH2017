\chapter{Resultados Esperados}

O projeto visa integrar WoTPy com grafos de conhecimento, buscando promover uma melhor interoperabilidade entre dispositivos IoT de diferentes fabricantes e potencializar a eficácia na comunicação e integração destes dispositivos. Neste contexto, os resultados esperados são multifacetados, abrangendo aspectos técnicos e aplicativos.

\section{Contribuições Técnicas}

\begin{itemize}
    \item \textbf{Integração Efetiva:} A conclusão bem-sucedida da integração do WoTPy com sistemas de grafos de conhecimento, resultando em uma solução robusta e estável.
    \item \textbf{Desenvolvimento de Endpoint SPARQL:} Implementação de um endpoint SPARQL eficiente, permitindo consultas dinâmicas sobre capacidades de sensores e suas observações.
    \item \textbf{Documentação Detalhada:} Disponibilização de documentação técnica abrangente, garantindo que os desenvolvedores possam entender, adotar e estender a solução proposta.
\end{itemize}

\section{Aplicações e Usabilidade}

\begin{itemize}
    \item \textbf{Interoperabilidade Aprimorada:} Dispositivos IoT de diferentes fabricantes poderão se comunicar e integrar de maneira mais eficiente, impulsionando a criação de sistemas IoT unificados.
    \item \textbf{Aplicações Semânticas:} A capacidade de explorar os benefícios da Web Semântica na IoT, possibilitando o desenvolvimento de aplicações mais inteligentes, personalizadas e contextuais.
    \item \textbf{Exemplo Prático:} Disponibilização de exemplo prático que demonstra o potencial e as capacidades da solução, facilitando sua adoção em projetos reais.
\end{itemize}

