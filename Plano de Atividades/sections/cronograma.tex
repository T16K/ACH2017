\chapter{Cronograma}

%
% https://www.overleaf.com/latex/examples/gantt-charts-with-the-pgfgantt-package/jmkwfxrnfxnw
% A fairly complicated example from section 2.9 of the package
% documentation. This reproduces an example from Wikipedia:
% http://en.wikipedia.org/wiki/Gantt_chart
%

\definecolor{barblue}{RGB}{153,204,254}
\definecolor{groupblue}{RGB}{51,102,254}
\definecolor{linkred}{RGB}{165,0,33}
\renewcommand\sfdefault{phv}
\renewcommand\mddefault{mc}
\renewcommand\bfdefault{bc}
\setganttlinklabel{s-s}{START-TO-START}
\setganttlinklabel{f-s}{FINISH-TO-START}
\setganttlinklabel{f-f}{FINISH-TO-FINISH}
\sffamily
\begin{adjustbox}{width=1.1\textwidth,center}
\begin{ganttchart}[
    canvas/.append style={fill=none, draw=black!5, line width=.75pt},
    hgrid style/.style={draw=black!5, line width=.75pt},
    vgrid={*1{draw=black!5, line width=.75pt}},
    today=4,
    today rule/.style={
      draw=black!64,
      dash pattern=on 3.5pt off 4.5pt,
      line width=1.5pt
    },
    today label font=\small\bfseries,
    title/.style={draw=none, fill=none},
    title label font=\bfseries\footnotesize,
    title label node/.append style={below=7pt},
    include title in canvas=false,
    bar label font=\mdseries\small\color{black!70},
    bar label node/.append style={left=2cm},
    bar/.append style={draw=none, fill=black!63},
    bar incomplete/.append style={fill=barblue},
    bar progress label font=\mdseries\footnotesize\color{black!70},
    group incomplete/.append style={fill=groupblue},
    group left shift=0,
    group right shift=0,
    group height=.5,
    group peaks tip position=0,
    group label node/.append style={left=.6cm},
    group progress label font=\bfseries\small,
    link/.style={-latex, line width=1.5pt, linkred},
    link label font=\scriptsize\bfseries,
    link label node/.append style={below left=-2pt and 0pt}
  ]{1}{12}
  \gantttitle[
    title label node/.append style={below left=7pt and -3pt}
  ]{MESES:\quad1}{1}
  \gantttitlelist{2,...,12}{1} \\
  
  \ganttgroup[progress=10]{WBS 1 Primeiro Semestre}{4}{7} \\
  \ganttbar[
    progress=30,
    name=WBS1A
  ]{\textbf{WBS 1.1} Analisar Especificações do W3C-WoT}{4}{4} \\
  \ganttbar[
    progress=40,
    name=WBS1B
  ]{\textbf{WBS 1.2} Analisar Dependências do WoTPy}{4}{4} \\
  \ganttbar[
    progress=0,
    name=WBS1C
  ]{\textbf{WBS 1.3} Resolver Problemas de Instalação}{5}{6} \\
  \ganttbar[
    progress=0,
    name=WBS1D
  ]{\textbf{WBS 1.4} Realizar Testes de Validação}{5}{6} \\
  \ganttbar[
    progress=0,
    name=WBS1E
  ]{\textbf{WBS 1.5} Desenvolver Exemplos de Uso}{6}{7} \\
  
  \ganttgroup[progress=0]{WBS 2 Segundo Semestre}{8}{12} \\
    \ganttbar[
    progress=0,
    name=WBS2A
  ]{\textbf{WBS 2.1} Capacitar em Grafos de Conhecimento}{8}{8} \\
  \ganttbar[
    progress=0,
    name=WBS2B
  ]{\textbf{WBS 2.2} Escolher Plataforma}{9}{9} \\
  \ganttbar[
    progress=0,
    name=WBS2C
  ]{\textbf{WBS 2.3} Armazenar Capacidades do Sensor e Observações}{9}{10} \\
  \ganttbar[
    progress=0,
    name=WBS2D
  ]{\textbf{WBS 2.4} Publicizar Grafos}{10}{12} \\
  \ganttbar[
    progress=0,
    name=WBS2E
  ]{\textbf{WBS 2.5} Definir e Implementar Casos de Teste}{12}{12} \\

  \ganttlink[link type=s-s]{WBS1A}{WBS1B}
  \ganttlink[link type=f-s]{WBS1B}{WBS1C}
  \ganttlink[link type=f-f]{WBS1C}{WBS1D}
  \ganttlink[link type=f-s]{WBS1D}{WBS1E}

  \ganttlink[link type=f-s]{WBS2A}{WBS2B}
  \ganttlink[link type=f-s]{WBS2B}{WBS2C}
  \ganttlink[link type=f-s]{WBS2C}{WBS2D}
  \ganttlink[link type=f-s]{WBS2D}{WBS2E}
  
  % \ganttlink[
    % link type=f-f,
    % link label node/.append style=left
  % {WBS1C}{WBS1D}
  
\end{ganttchart}
\end{adjustbox}

%
% A simpler example from the package documentation:
%

% \begin{ganttchart}{1}{12}
  % \gantttitle{2011}{12} \\
  % \gantttitlelist{1,...,12}{1} \\
  % \ganttgroup{Group 1}{1}{7} \\
  % \ganttbar{Task 1}{1}{2} \\
  % \ganttlinkedbar{Task 2}{3}{7} \ganttnewline
  % \ganttmilestone{Milestone}{7} \ganttnewline
  % \ganttbar{Final Task}{8}{12}
  % \ganttlink{elem2}{elem3}
  % \ganttlink{elem3}{elem4}
% \end{ganttchart}

% \end{document}

No início do primeiro semestre, uma análise detalhada das especificações do W3C-WoT será realizada (\ref{estudo}), seguida pela seleção e estudo das bibliotecas e ferramentas relevantes para o projeto (\ref{dependência}) Em paralelo, serão identificados e resolvidos problemas de instalação do WoTPy (\ref{instalação}). Com as melhorias implementadas, serão realizados testes e validações para garantir a qualidade e a funcionalidade das contribuições feitas (\ref{teste}).

No final do primeiro semestre, serão desenvolvidos exemplos de uso e documentação detalhada do projeto (\ref{uso}), auxiliando desenvolvedores e usuários na implementação do WoTPy em seus próprios projetos IoT. O exemplo prático incluirá ajustes na comunicação do dispositivo Protetor Solar (ou similar).

Para o segundo semestre, o que se espera inclui a capacitação em ferramentas de grafos de conhecimento, a escolha de uma plataforma de armazenamento para esses grafos, o armazenamento das capacidades do sensor em SSN e das observações em SOSA, a publicização desses grafos por meio de um endpoint SPARQL e a definição e implementação de casos de teste para a integração.

