\chapter{Objetivos}

\section{Objetivo Geral}

% (Descrição do objetivo geral do trabalho.)

O objetivo geral deste trabalho é contribuir para o desenvolvimento do WoTPy \url{https://github.com/agmangas/wot-py}, um \textit{gateway} experimental baseado no W3C-WoT, que melhora a interoperabilidade entre dispositivos IoT de diferentes fabricantes, facilitando a comunicação e integração em uma única aplicação ou sistema.

\section{Objetivo Específico}

% (Descrição dos objetivos específicos do trabalho.)

Para atingir o objetivo geral, os seguintes objetivos específicos foram estabelecidos:

\begin{enumerate}
    \item Analisar as especificações do W3C-WoT \cite{WoTArchitecture} e compreender os principais conceitos e requisitos para a implementação do WoTPy;
    \item Selecionar e estudar as bibliotecas e ferramentas das dependências do WoTPy;
    \item Resolver problemas de instalação, garantindo que o WoTPy possa ser facilmente implantado e configurado em diferentes ambientes e sistemas;
    \item Testar e validar as contribuições feitas;
    \item Desenvolver exemplo de uso e documentação detalhada para auxiliar desenvolvedores e usuários na implementação do WoTPy em seus projetos IoT;
\end{enumerate}

Caso esses objetivos específicos sejam cumpridos no primeiro semestre de 2023, o objetivo para o próximo semestre (segundo semestre de 2023) é abordar a integração do WoTPy com grafos de conhecimento. Essa integração visa melhorar ainda mais a interoperabilidade dos dispositivos IoT e possibilitar a divulgação das capacidades dos sensores e suas observações através de um endpoint SPARQL.

