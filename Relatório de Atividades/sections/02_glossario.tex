% ----------------------------------------------------------
% Glossário
% ----------------------------------------------------------
%
% Consulte o manual da classe abntex2 para orientações sobre o glossário.
%
%\glossary



% Aqui as palavras aparecerão em ordem alfabética. A palavra ou sigla a ser definida aparecerá em negrito seguida de dois pontos (:), e em seguida a definição é escrita sem negrito. Ex:
%
% GLCE: Gramática Livre de Contexto Estocástica – gramática livre de contexto com uma distribuição de probabilidades sobre as produções com o mesmo lado esquerdo.
%
% No caso de abreviaturas (siglas), mesmo descrevendo-as aqui não deixe de defini-las na primeira vez em que elas são empregadas!

\newglossaryentry{WoTPy}{
name={WoTPy},
description={WoTPy é um gateway experimental baseado no W3C-WoT (Web of Things). Ele visa melhorar a interoperabilidade e a integração de dispositivos IoT de diferentes fabricantes, facilitando a comunicação e a interconexão em uma única aplicação ou sistema.}
}

\newglossaryentry{Gateway}{
name={Gateway},
description={Um gateway é um dispositivo ou software que conecta redes ou sistemas diferentes, permitindo a comunicação e a transferência de dados entre eles.}
}

\newglossaryentry{iot}{
name={IoT},
description={IoT (Internet of Things) refere-se à interconexão de dispositivos físicos (coisas) que são capazes de coletar e trocar dados por meio de uma rede.}
}

\newglossaryentry{protocolos de comunicação}{
name={protocolos de comunicação},
description={Protocolos de comunicação são conjuntos de regras e formatos de dados que governam a troca de informações entre sistemas ou dispositivos de rede.}
}

\newglossaryentry{grafos de conhecimento}{
name={grafos de conhecimento},
description={Grafos de conhecimento são estruturas de dados que representam conhecimento e informações em forma de grafos, onde os nós representam entidades e as arestas representam as relações entre elas.}
}

\newglossaryentry{endpoint SPARQL}{
name={endpoint SPARQL},
description={Um endpoint SPARQL é uma interface de acesso a um grafo de conhecimento que permite consultas e recuperação de informações usando a linguagem de consulta SPARQL.}
}

\newglossaryentry{interoperabilidade}{
name={interoperabilidade},
description={Interoperabilidade refere-se à capacidade de diferentes sistemas ou dispositivos se comunicarem, trocarem informações e trabalharem juntos de forma eficiente e eficaz.}
}

\newglossaryentry{interconexão}{
name={interconexão},
description={A interconexão se refere à conexão e integração de diferentes sistemas, dispositivos ou redes para permitir a comunicação e a troca de informações entre eles.}
}

\newglossaryentry{w3c}{
name={W3C},
description={O W3C (World Wide Web Consortium) é um consórcio internacional que desenvolve padrões e diretrizes para a World Wide Web, visando promover sua acessibilidade, usabilidade e interoperabilidade.}
}

\newglossaryentry{wot}{
name={WoT},
description={WoT (Web of Things) é um termo que se refere à extensão da Web para abranger a interconexão e interação de dispositivos físicos por meio de padrões e tecnologias da Web.}
}

\newglossaryentry{Descrição de Coisas (Thing Descriptions)}{
name={Descrição de Coisas (Thing Descriptions)},
description={Thing Descriptions são descrições de metadados e interfaces de rede de coisas (dispositivos) no contexto do W3C-WoT. Elas fornecem informações sobre as capacidades, propriedades e serviços oferecidos pelas coisas.}
}

\newglossaryentry{Modelos de Vinculação (Binding Templates)}{
name={Modelos de Vinculação (Binding Templates)},
description={Binding Templates são modelos que fornecem orientações para definir interfaces de rede para protocolos específicos e ecossistemas de IoT no contexto do W3C-WoT. Eles ajudam a garantir a compatibilidade e a integração entre diferentes sistemas.}
}

\newglossaryentry{Scripting API}{
name={Scripting API},
description={Scripting API é uma API (Application Programming Interface) baseada em JavaScript que permite a implementação da lógica de aplicação das coisas no contexto do W3C-WoT. Ela simplifica o desenvolvimento de aplicativos IoT e promove a portabilidade entre fornecedores e dispositivos.}
}

\newglossaryentry{arquitetura}{
name={arquitetura},
description={A arquitetura refere-se à estrutura geral e organização de um sistema ou aplicação, incluindo seus componentes, padrões e princípios subjacentes.}
}

\newglossaryentry{web}{
name={web},
description={A Web (World Wide Web) é um sistema de informação e documentos interconectados, acessíveis por meio da Internet e navegadores da web.}
}

\newglossaryentry{microcontrolador}{
name={microcontrolador},
description={Um microcontrolador é um dispositivo eletrônico que incorpora um microprocessador, memória e periféricos em um único chip. Ele é projetado para controlar funções específicas em sistemas eletrônicos.}
}

\newglossaryentry{servidor}{
name={servidor},
description={Um servidor é um computador ou sistema que fornece serviços, recursos ou funcionalidades a outros dispositivos ou sistemas, conhecidos como clientes, por meio de uma rede.}
}

\newglossaryentry{handlers}{
name={handlers},
description={Handlers são componentes de software responsáveis por lidar com solicitações, eventos ou tarefas específicas em um sistema ou aplicação. Eles são responsáveis por receber, processar e responder a requisições ou ações específicas.}
}

\glsaddall
\printglossary

\cleardoublepage
