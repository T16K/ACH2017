% resumo em inglês
%-------------------------------------------------------------------------
% Comentário adicional do PPgSI - Informações sobre ``resumo em inglês''
% 
% Caso a Qualificação ou a Dissertação/Tese inteira seja elaborada no idioma inglês, 
% então o ``Abstract'' vem antes do ``Resumo''.
% 
%-------------------------------------------------------------------------
\begin{resumo}[Abstract]
\begin{otherlanguage*}{english}

%-------------------------------------------------------------------------
% Comentário adicional do PPgSI - Informações sobre ``referência em inglês''
% 
% Troque os seguintes campos pelos dados de sua Dissertação/Tese (mantendo a 
% formatação e pontuação):
%     - SURNAME
%     - FirstName1
%     - MiddleName1
%     - MiddleName2
%     - Work title: work subtitle
%     - DefenseYear (Ano de Defesa)
%
% Mantenha todas as demais informações exatamente como estão.
%
% [Não usar essas informações de ``referência'' para Qualificação]
%
%-------------------------------------------------------------------------
\begin{flushleft}
ARIGA, Gustavo Tsuyoshi. \textbf{WotPy: Troubleshooting and Usage Example.} \imprimirdata. \pageref{LastPage} p. Activity Report – School of Arts, Sciences and Humanities, University of São Paulo, São Paulo, 2023. 
\end{flushleft}

The purpose of this study was to contribute to the development of WoTPy, an experimental gateway based on W3C-WoT, with the aim of improving interoperability among IoT (Internet of Things) devices from different manufacturers. To achieve this goal, specific objectives were established, including the analysis of W3C-WoT specifications, the selection and study of libraries and tools related to WoTPy dependencies, addressing installation challenges, conducting tests and validating contributions, as well as developing practical examples and detailed documentation. Understanding the specifications enabled the correct and efficient implementation of WoTPy, while the selection and mastery of appropriate libraries and tools ensured support for communication protocols and ease of integration. Resolving installation problems made WoTPy easily deployable and configurable in various environments and operating systems, facilitating its adoption by developers and end users. The developed usage examples and detailed documentation assisted other developers and users in implementing WoTPy in their IoT projects. In conclusion, WoTPy is a functional and effective gateway, promoting interoperability and integration of IoT devices. For future advancements, the integration of WoTPy with knowledge graphs will be explored, enabling the presentation of sensor capabilities and observations through a SPARQL endpoint, expanding the options for connectivity and data sharing among IoT devices. This research has contributed to the advancement of the Internet of Things field, providing a pragmatic and effective solution to enhance communication and unification of IoT devices.

Keywords: Web das Coisas (WoT). World Wide Web Consortium (W3C). WoTPy.
\end{otherlanguage*}
\end{resumo}