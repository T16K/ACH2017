\chapter{Objetivos}

\section{Objetivo Geral}

o objetivo geral deste trabalho foi contribuir para o desenvolvimento do WoTPy \citeonline{gitwotpy:2022}, um \textit{gateway} experimental baseado no W3C-WoT, que melhora a interoperabilidade entre dispositivos IoT de diferentes fabricantes, facilitando a comunicação e integração em uma única aplicação ou sistema.

\section{Objetivo Específico}

para atingir o objetivo geral, os seguintes objetivos específicos foram estabelecidos:

\begin{enumerate}
    \item Analisar as especificações do W3C-WoT \citeonline{Architecture} e compreender os principais conceitos e requisitos para a implementação do WoTPy;
    \item Selecionar e estudar as bibliotecas e ferramentas das dependências do WoTPy;
    \item Resolver problemas de instalação, garantindo que o WoTPy possa ser facilmente implantado e configurado em diferentes ambientes e sistemas;
    \item Testar e validar as contribuições feitas;
    \item Desenvolver exemplo de uso e documentação detalhada para auxiliar desenvolvedores e usuários na implementação do WoTPy em seus projetos IoT;
\end{enumerate}

\section{Extensão}

com esses objetivos específicos alcançados no primeiro semestre de 2023, pretende-se abordar, no segundo semestre de 2023, a integração do wotpy com grafos de conhecimento. Essa integração busca aprimorar ainda mais a interoperabilidade dos dispositivos IoT, possibilitando a divulgação das capacidades dos sensores e suas observações por meio de um \textit{endpoint SPARQL}.



