\chapter{Discussão}

A seção de discussão tem como objetivo analisar e interpretar os resultados obtidos no trabalho, relacionando-os com a literatura existente e abordando os principais pontos e contribuições do estudo.

\section{Análise dos Resultados}

Os resultados obtidos demonstraram que o WoTPy é uma solução viável para a implementação de um \textit{gateway} IoT baseado no W3C-WoT. Através da seleção cuidadosa das bibliotecas e ferramentas adequadas, foi possível desenvolver um gateway que suporta os principais protocolos de comunicação, como HTTP. Isso permite a integração de dispositivos IoT de diferentes fabricantes e facilita a interoperabilidade entre eles.

A compreensão das especificações do W3C-WoT foi essencial para o desenvolvimento do WoTPy. O estudo detalhado das especificações, como as Descrição das Coisas, Binding Templates e a \textit{Scripting API}, permitiu a implementação correta e eficiente do \textit{gateway}, garantindo a conformidade com os padrões estabelecidos.

A resolução dos problemas de instalação do WoTPy também se mostrou crucial. A simplificação do processo de instalação e configuração contribuiu para a fácil adoção do \textit{gateway} em diferentes ambientes e sistemas operacionais. Além disso, os testes e validação realizados confirmaram a qualidade e a funcionalidade do WoTPy, fornecendo confiabilidade e confiança na sua utilização.

\section{Comparação com Trabalhos Relacionados}

Ao comparar o WoTPy com outros \textit{gateways} IoT existentes, podemos observar suas vantagens e contribuições específicas. O WoTPy destaca-se por sua conformidade com as especificações do W3C-WoT, o que garante a interoperabilidade e a padronização no contexto da Web das Coisas. Além disso, sua flexibilidade e suporte aos principais protocolos de comunicação o tornam uma opção atrativa para a integração de dispositivos IoT heterogêneos.

A documentação detalhada e os exemplos de uso desenvolvidos também são diferenciais importantes do WoTPy. Esses recursos facilitam a compreensão e a utilização do gateway por parte dos desenvolvedores e usuários, contribuindo para a disseminação e adoção do projeto.

\section{Limitações e Possíveis Melhorias}

Durante o desenvolvimento deste trabalho, uma limitação identificada foi a incapacidade de utilizar o WoTPy no MicroPython. O MicroPython é uma implementação leve e eficiente do Python projetada para rodar em dispositivos com recursos limitados, como microcontroladores. No entanto, devido a diferenças de recursos e suporte a bibliotecas, o WoTPy não é atualmente compatível com o MicroPython.

Uma possível melhoria para contornar essa limitação seria a adaptação do WoTPy para ser compatível com o MicroPython. Isso permitiria que o gateway fosse implantado em dispositivos com recursos restritos, ampliando seu alcance e possibilitando sua utilização em uma variedade maior de cenários IoT. Essa adaptação envolveria ajustes nas dependências e otimizações específicas para o ambiente do MicroPython.

Além disso, uma outra melhoria potencial seria explorar alternativas de implementação específicas para o MicroPython. Isso poderia envolver a criação de uma versão simplificada do WoTPy ou o desenvolvimento de um gateway específico para dispositivos MicroPython. Essas abordagens permitiriam um melhor aproveitamento dos recursos limitados do MicroPython, garantindo a compatibilidade e a eficiência do gateway em tais dispositivos.

Essas melhorias seriam importantes para ampliar o alcance do WoTPy e torná-lo mais acessível a um maior número de dispositivos IoT, incluindo aqueles baseados no MicroPython. Ao superar a limitação atual e fornecer suporte para o MicroPython, o WoTPy se tornaria uma solução mais abrangente e versátil para a integração de dispositivos IoT em diferentes plataformas e ambientes.

\section{Contribuições e Impacto}

O presente trabalho contribui para o avanço da interoperabilidade e integração de dispositivos IoT no contexto da Web das Coisas. O desenvolvimento e a implementação do WoTPy como um gateway baseado no W3C-WoT oferecem uma solução prática e padronizada para a comunicação entre dispositivos de diferentes fabricantes.

As contribuições deste trabalho são relevantes tanto para a academia quanto para a indústria. Os resultados obtidos podem servir como base para futuras pesquisas e desenvolvimentos na área de IoT. Além disso, o WoTPy pode ser utilizado por empresas e profissionais que buscam criar soluções IoT interoperáveis e eficientes.

\section{Considerações Finais}

Através da metodologia adotada e da análise dos resultados, foi possível constatar que o WoTPy é uma solução eficaz para a implementação de um gateway IoT baseado no W3C-WoT. Suas funcionalidades, conformidade com as especificações do W3C e facilidade de uso o tornam uma opção viável e promissora no contexto da Web das Coisas.

O trabalho realizado contribui para a disseminação e adoção do WoTPy, bem como para a melhoria da interoperabilidade e integração de dispositivos IoT. Espera-se que as limitações identificadas possam ser superadas e que as melhorias sugeridas possam ser implementadas em trabalhos futuros.
