% ---
% RESUMOS
% ---

% resumo em português
\setlength{\absparsep}{18pt} % ajusta o espaçamento dos parágrafos do resumo
\begin{resumo}

%-------------------------------------------------------------------------
% Comentário adicional do PPgSI - Informações sobre ``referência'':
% 
% Troque os seguintes campos pelos dados de sua Dissertação/Tese (mantendo a 
% formatação e pontuação):
%   - SOBRENOME
%   - Nome1
%   - Nome2
%   - Nome3
%   - Título do trabalho: subtítulo do trabalho
%   - AnoDeDefesa
%
% Mantenha todas as demais informações exatamente como estão.
% 
% [Não usar essas informações de ``referência'' para Qualificação]
%
% Para Tese de Doutorado: trocar "Dissertação (Mestrado em Ciências)" por "Tese (Doutorado em Ciências)".
%-------------------------------------------------------------------------
\begin{flushleft}
ARIGA, Gustavo Tsuyoshi. \textbf{WotPy: Solução de Problemas e Exemplo de Uso}. \imprimirdata. \pageref{LastPage} p. Relatório de Atividades – Escola de Artes, Ciências e Humanidades, Universidade de São Paulo, São Paulo, 2023. \end{flushleft}

% Parágrafo único e no máximo 500 palavras. constituído de uma sequência de frases concisas e objetivas, em forma de texto. Deve apresentar os objetivos, métodos empregados, resultados e conclusões.

O propósito deste estudo foi auxiliar no progresso do WoTPy, um \textit{gateway} experimental fundado no W3C-WoT, com a intenção de aprimorar a comunicação entre aparelhos IoT (Internet das Coisas) de diversas marcas. Para atingir tal propósito, metas específicas foram estipuladas, como o exame das normas do W3C-WoT, a escolha e investigação das bibliotecas e utensílios ligados às dependências do WoTPy, o enfrentamento de desafios de instalação, a execução de testes e a confirmação da validade das colaborações, além da elaboração de exemplos práticos e documentação detalhada. A compreensão das especificações possibilitou sua implementação correta e eficiente, enquanto a seleção e domínio das bibliotecas e ferramentas adequadas garantiram suporte aos protocolos de comunicação e facilidade de integração. A resolução dos problemas de instalação tornou o WoTPy facilmente implantável e configurável em diferentes ambientes e sistemas operacionais, favorecendo sua adoção por desenvolvedores e usuários finais. Os exemplos de uso e a documentação detalhada desenvolvidos auxiliaram outros desenvolvedores e usuários na implementação do WoTPy em seus projetos IoT. Conclui-se, portanto, que o WoTPy é um \textit{gateway} funcional e eficaz, promovendo a interoperabilidade e integração de dispositivos IoT. Para futuros avanços, a junção do WoTPy com grafos de conhecimento será avaliada, o que permite a apresentação das habilidades dos sensores e suas observações através de um \textit{endpoint SPARQL}, expandindo as opções de conexão e compartilhamento de dados entre aparelhos IoT. Esta pesquisa auxiliou na evolução do campo da Internet das Coisas, apresentando uma alternativa pragmática e eficaz para aprimorar a comunicação e unificação de dispositivos IoT.

Palavras-chaves: Web das Coisas (WoT). World Wide Web Consortium (W3C). WoTPy.
\end{resumo}

% (pelo menos 3 e no máximo 5, lembrando que um conjunto de palavras pode formar o conceito que chamamos de palavras chave, por exemplo, "Sistemas de Informação", constitui UMA palavra chave).
