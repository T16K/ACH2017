% ---
% RESUMOS
% ---

% resumo em português
\setlength{\absparsep}{18pt} % ajusta o espaçamento dos parágrafos do resumo
\begin{resumo}

%-------------------------------------------------------------------------
% Comentário adicional do PPgSI - Informações sobre ``referência'':
% 
% Troque os seguintes campos pelos dados de sua Dissertação/Tese (mantendo a 
% formatação e pontuação):
%   - SOBRENOME
%   - Nome1
%   - Nome2
%   - Nome3
%   - Título do trabalho: subtítulo do trabalho
%   - AnoDeDefesa
%
% Mantenha todas as demais informações exatamente como estão.
% 
% [Não usar essas informações de ``referência'' para Qualificação]
%
% Para Tese de Doutorado: trocar "Dissertação (Mestrado em Ciências)" por "Tese (Doutorado em Ciências)".
%-------------------------------------------------------------------------
\begin{flushleft}
ARIGA, Gustavo Tsuyoshi. \textbf{WotPy: Solução de Problemas e Exemplo de Uso}. \imprimirdata. \pageref{LastPage} Relatório de Atividades – Escola de Artes, Ciências e Humanidades, Universidade de São Paulo, São Paulo, 2023. \end{flushleft}

% Parágrafo único e no máximo 500 palavras. constituído de uma sequência de frases concisas e objetivas, em forma de texto. Deve apresentar os objetivos, métodos empregados, resultados e conclusões.

Este trabalho teve como objetivo contribuir para o desenvolvimento do WoTPy, um \textit{gateway} experimental baseado no W3C-WoT, visando melhorar a interoperabilidade entre dispositivos IoT de diferentes fabricantes. Para alcançar esse objetivo, foram estabelecidos objetivos específicos, incluindo a análise das especificações do W3C-WoT, a seleção e estudo das bibliotecas e ferramentas relacionadas às dependências do WoTPy, a resolução de problemas de instalação, a realização de testes e validação das contribuições, e o desenvolvimento de exemplos de uso e documentação detalhada. Utilizando métodos de pesquisa bibliográfica, análise documental e experimentação, os resultados obtidos demonstraram que o WoTPy é capaz de facilitar a comunicação e integração entre dispositivos IoT, seguindo as especificações do W3C-WoT. A compreensão das especificações possibilitou sua implementação correta e eficiente, enquanto a seleção e domínio das bibliotecas e ferramentas adequadas garantiram suporte aos protocolos de comunicação e facilidade de integração. A resolução dos problemas de instalação tornou o WoTPy facilmente implantável e configurável em diferentes ambientes e sistemas operacionais, favorecendo sua adoção por desenvolvedores e usuários finais. Os exemplos de uso e a documentação detalhada desenvolvidos auxiliaram outros desenvolvedores e usuários na implementação do WoTPy em seus projetos IoT. Conclui-se, portanto, que o WoTPy é um \textit{gateway} funcional e eficaz, promovendo a interoperabilidade e integração de dispositivos IoT. Como direcionamento para desenvolvimentos futuros, sugere-se a integração do WoTPy com grafos de conhecimento, possibilitando a divulgação das capacidades dos sensores e suas observações por meio de um \textit{endpoint SPARQL}, ampliando as possibilidades de integração e troca de informações entre dispositivos IoT. Este trabalho contribuiu para o avanço da área de Internet das Coisas, oferecendo uma solução prática e eficiente para melhorar a interoperabilidade e integração de dispositivos IoT.

Palavras-chaves: Web das Coisas (WoT). World Wide Web Consortium (W3C). WoTPy.
\end{resumo}

% (pelo menos 3 e no máximo 5, lembrando que um conjunto de palavras pode formar o conceito que chamamos de palavras chave, por exemplo, "Sistemas de Informação", constitui UMA palavra chave).
