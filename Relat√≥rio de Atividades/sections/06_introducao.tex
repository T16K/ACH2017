\chapter{Introdução}

Neste capítulo, serão apresentados a contextualização, a justificativa/problema de pesquisa, a hipótese/proposição, os objetivos e o método de pesquisa do projeto. Além disso, será feita uma breve visão geral da estrutura do documento.

\section{Contextualização e Motivação}

Este projeto de pesquisa está inserido na área de Internet das Coisas (IoT), que se refere à interconexão de dispositivos inteligentes em uma rede global. A IoT tem o potencial de transformar diversos setores, como saúde, indústria, agricultura e cidades inteligentes, trazendo benefícios como automação, eficiência energética e monitoramento remoto.

No entanto, a interoperabilidade entre dispositivos IoT de diferentes fabricantes ainda é um desafio significativo. A falta de padronização e a diversidade de protocolos e interfaces dificultam a comunicação e integração entre esses dispositivos. Isso limita a capacidade de criar soluções abrangentes e escaláveis que aproveitem todo o potencial da IoT.

Diante desse cenário, o projeto de pesquisa propõe abordar o problema da interoperabilidade, focando na implementação e melhoria do WoTPy, um \textit{gateway} experimental baseado no W3C-WoT. O objetivo é desenvolver uma solução que facilite a comunicação e a integração entre dispositivos IoT de diferentes fabricantes, seguindo os padrões e diretrizes estabelecidos pelo W3C.

\section{Justificativa/Problema de Pesquisa}

A justificativa para a realização deste projeto de pesquisa reside na necessidade de superar as barreiras de interoperabilidade na IoT. A falta de padronização e a diversidade de protocolos e interfaces dificultam a comunicação e a integração entre os dispositivos IoT, limitando o potencial da tecnologia.

O problema de pesquisa consiste em desenvolver uma solução que promova a interoperabilidade entre dispositivos IoT de diferentes fabricantes. O objetivo é permitir que esses dispositivos se comuniquem e interajam de forma transparente, possibilitando a criação de soluções abrangentes e escaláveis na IoT.

\section{Hipótese/Proposição}

Com base na análise do problema de pesquisa, é proposta a hipótese de que a implementação e melhoria do WoTPy como um \textit{gateway} experimental baseado no W3C-WoT pode facilitar a comunicação e a integração entre dispositivos IoT de diferentes fabricantes. Acredita-se que essa solução, alinhada aos padrões e diretrizes estabelecidos pelo W3C, pode contribuir para a superação das barreiras de interoperabilidade na IoT.

\section{Objetivos}

O objetivo geral deste projeto de pesquisa é desenvolver e aprimorar o WoTPy como um \textit{gateway} experimental baseado no W3C-WoT, visando melhorar a interoperabilidade entre dispositivos IoT de diferentes fabricantes. Para atingir esse objetivo geral, foram estabelecidos os seguintes objetivos específicos:

\begin{itemize}
    \item Compreender as especificações do W3C-WoT e suas principais componentes, como Descrição das Coisas (\textit{Thing Descriptions}), Modelos de Vinculação (\textit{Binding Templates}) e \textit{Scripting API}.
    \item Selecionar e estudar as bibliotecas e ferramentas necessárias para o desenvolvimento do WoTPy.
    \item Resolver problemas de instalação e configuração do WoTPy, visando facilitar sua implantação em diferentes ambientes.
    \item Realizar testes e validações para garantir a funcionalidade e qualidade do WoTPy.
    \item Desenvolver exemplos de uso práticos do WoTPy e documentação detalhada para auxiliar desenvolvedores e usuários na implementação e utilização do \textit{gateway}.
\end{itemize}

\section{Método de Pesquisa}

Este projeto de pesquisa adota uma abordagem de pesquisa aplicada, combinando elementos de pesquisa exploratória e desenvolvimento de \textit{software}. A metodologia incluiu as seguintes etapas:

\begin{enumerate}
    \item Revisão bibliográfica sobre o W3C-WoT, IoT e interoperabilidade.
    \item Análise das especificações do W3C-WoT e estudo aprofundado de suas principais componentes.
\item Seleção e estudo das bibliotecas e ferramentas relacionadas   ao WoTPy.
    \item Desenvolvimento e aprimoramento do WoTPy, incluindo resolução de problemas de instalação, testes e validações.
    \item Criação de exemplos de uso práticos do WoTPy e elaboração de documentação detalhada.
\end{enumerate}

\section{Estrutura do Documento}

Este documento está estruturado da seguinte forma:

\begin{itemize}
    \item Capítulo 1: Introdução - apresenta a contextualização, a justificativa, a hipótese, os objetivos e o método de pesquisa do projeto.
    \item Capítulo 2: Objetivos - apresenta os objetivos gerais e específicos do projeto.
    \item Capítulo 3: Revisão Bibliográfica - aborda os conceitos fundamentais relacionados ao W3C-WoT, IoT e interoperabilidade.
    \item Capítulo 4: Metodologia - descreve detalhadamente as etapas e os procedimentos adotados no desenvolvimento do projeto.
    \item Capítulo 5: Resultados - apresenta os resultados obtidos durante a implementação e aprimoramento do WoTPy.
    \item Capítulo 6: Discussão - analisa e discute os resultados obtidos, destacando suas contribuições e limitações.
    \item Capítulo 7: Conclusão - sintetiza os principais pontos abordados no trabalho, apresenta as conclusões alcançadas e sugere possíveis direções futuras.
    \item Apêndice A: relata as dificuldades e os problemas enfrentados durante o desenvolvimento do trabalho.
\end{itemize}
