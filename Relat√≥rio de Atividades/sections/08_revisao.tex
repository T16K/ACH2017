
\chapter{Revisão Bibliográfica}

Neste capítulo, será realizado uma análise de literatura referente à Web das Coisas (WoT) e sua estrutura sugerida pelo World Wide Web Consortium (W3C). Também será discutido os desafios envolvidos na interoperabilidade entre redes IoT e o papel dos \textit{gateways} neste cenário.

\section{A Web das Coisas e a Estrutura do W3C}

A Web das Coisas (WoT) é um conceito que tem recebido atenção crescente recentemente. O W3C descreve a WoT como a conexão de sensores, atuadores, objetos físicos, locais e até pessoas por meio da Internet. O intuito da WoT é usar as tecnologias da Web para simplificar o desenvolvimento de aplicações e serviços que envolvam esses dispositivos e suas representações virtuais \cite{WoTTerminology, WoTCommunityWiki}.

A estrutura proposta pelo W3C para a WoT determina os elementos que compõem esta rede interligada. Ela ressalta a relevância da infraestrutura e dos protocolos da Internet para possibilitar a comunicação entre os dispositivos. Ademais, a estrutura enfatiza a existência de dispositivos de borda, também conhecidos como \textit{gateways}, que exercem um papel crucial na interoperabilidade entre as redes IoT e a Web \cite{Matsukura:23:WTA}.

A estrutura da WoT define os dispositivos de borda como pontos de conexão entre a rede interna e a Internet. Estes dispositivos são encarregados de realizar a computação e o processamento de dados perto dos sensores e atuadores, frequentemente em ambientes com recursos limitados. Este método, conhecido como computação de borda ou \textit{fog computing}, permite a implementação de soluções de interoperabilidade e segurança no nível do \textit{gateway} \cite{Stirbu2008, Gyrard2017, GARCIAMANGAS2019235}.

\section{O Framework WoTPy}

Dentro do contexto da WoT, o WoTPy surge como um framework desenvolvido no meio acadêmico com a finalidade de implementar as recomendações do W3C. O WotPy é um conjunto de ferramentas que permite o desenvolvimento de \textit{gateways} IoT, englobando diferentes protocolos de comunicação, como HTTP, MQTT e CoAP. Ele foi concebido para facilitar a interoperabilidade entre dispositivos e oferecer suporte a funcionalidades cruciais, como privacidade, segurança, descoberta de dispositivos e interfaces de usuário \cite{GARCIAMANGAS2019235}.