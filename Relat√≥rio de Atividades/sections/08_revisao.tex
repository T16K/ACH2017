\chapter{Revisão Bibliográfica}

Neste capítulo, será apresentada uma revisão bibliográfica sobre a Web das Coisas (WoT) e sua arquitetura proposta pelo World Wide Web Consortium (W3C). Também serão discutidos os desafios relacionados à interoperabilidade entre redes IoT e o papel dos \textit{gateways} nesse contexto. 

\section{Web das Coisas e a Arquitetura Proposta pelo W3C}

A Web das Coisas (WoT) é um conceito que tem ganhado destaque nos últimos anos. O W3C define a WoT como a interconexão de sensores, atuadores, objetos físicos, locais e até mesmo pessoas por meio da Internet. O objetivo da WoT é utilizar as tecnologias Web para facilitar o desenvolvimento de aplicações e serviços que envolvam esses dispositivos e suas representações virtuais \cite{WoTTerminology, WoTCommunityWiki}.

A arquitetura proposta pelo W3C para a WoT define os elementos que compõem essa rede interconectada. Ela destaca a importância da infraestrutura e dos protocolos da Internet para viabilizar a comunicação entre os dispositivos. Além disso, a arquitetura enfatiza a existência de dispositivos de borda, também conhecidos como \textit{gateways}, que desempenham um papel fundamental na interoperabilidade entre as redes IoT e a \textit{Web} \cite{Matsukura:23:WTA}.

A arquitetura da WoT define os dispositivos de borda como pontos de interconexão entre a rede interna e a \textit{Internet}. Esses dispositivos são responsáveis por realizar a computação e o processamento de dados próximo aos sensores e atuadores, muitas vezes em ambientes com recursos limitados. Essa abordagem, conhecida como computação de borda ou \textit{fog computing}, permite a implementação de soluções de interoperabilidade e segurança no nível do \textit{gateway} \cite{Stirbu2008, Gyrard2017, GARCIAMANGAS2019235}.

\section{O Arcabouço WoTPy}

No contexto da WoT, o WoTPy se destaca como um arcabouço desenvolvido no ambiente acadêmico com o objetivo de implementar as recomendações do W3C. O WotPy é um conjunto de ferramentas que permite o desenvolvimento de \textit{gateways} IoT, abrangendo diferentes protocolos de comunicação, como HTTP, MQTT e CoAP. Ele foi projetado para simplificar a interoperabilidade entre dispositivos e oferecer suporte a funcionalidades essenciais, como privacidade, segurança, descoberta de dispositivos e interfaces de usuário \cite{GARCIAMANGAS2019235}.

