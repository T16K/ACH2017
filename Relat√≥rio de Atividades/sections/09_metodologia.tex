\chapter{Metodologia}

A metodologia utilizada para a realização deste trabalho, apresentará as etapas e abordagens adotadas. A metodologia foi planejada com o objetivo de alcançar os resultados esperados e responder às questões de pesquisa estabelecidas.

\section{Abordagem de Pesquisa}

A abordagem de pesquisa adotada neste trabalho foi baseada em uma combinação de pesquisa exploratória, pesquisa bibliográfica e desenvolvimento prático. A pesquisa exploratória foi realizada para obter uma compreensão aprofundada do tema, identificar os principais conceitos e tendências, e explorar as soluções existentes. A pesquisa bibliográfica foi conduzida para revisar e analisar as principais publicações, artigos científicos e padrões relacionados ao WoTPy e ao W3C-WoT. Isso permitiu embasar teoricamente o trabalho e identificar lacunas ou oportunidades de melhoria. O desenvolvimento prático envolveu a implementação e o teste do WoTPy e a criação de exemplos de uso e documentação detalhada.

\section{Coleta de Dados}

A coleta de dados foi realizada por meio de diversas fontes, incluindo artigos científicos, publicações, documentação oficial do W3C-WoT e WoTPy, bem como experimentações práticas. A revisão bibliográfica foi conduzida para coletar informações relevantes sobre o W3C-WoT, as tecnologias envolvidas, os padrões e as melhores práticas. A documentação oficial do W3C-WoT \citeonline{Architecture} e do WoTPy \citeonline{wotpy} foi consultada para compreender as especificações, APIs e diretrizes recomendadas. 

\section{Desenvolvimento do WoTPy}

O desenvolvimento do WoTPy foi conduzido com base nas especificações e diretrizes do W3C-WoT. Inicialmente, foi realizada uma análise detalhada das especificações do W3C-WoT para compreender os principais conceitos e requisitos. Em seguida, foram selecionadas as bibliotecas e ferramentas adequadas como dependências do WoTPy. Foram resolvidos problemas de instalação e configuração para garantir a facilidade de implantação em diferentes ambientes e sistemas operacionais. A implementação do WoTPy foi realizada utilizando a linguagem de programação Python e seguindo as boas práticas de desenvolvimento de \textit{software}.

\section{Validação e Testes}

A validação do WoTPy foi realizada por meio de um conjunto de testes disponíveis no repositório wot-py \citeonline{gitwotpy:2022}. Os testes foram projetados para verificar a funcionalidade, a compatibilidade e a interoperabilidade do WoTPy. Foram realizados testes unitários, testes de integração e testes de sistema para garantir a qualidade e a robustez do WoTPy. Os resultados dos testes foram analisados e registrados para avaliar o desempenho e identificar possíveis problemas ou melhorias.

\section{Desenvolvimento do Exemplo de Uso e Documentação}

Para auxiliar os desenvolvedores e usuários na implementação do WoTPy em seus projetos IoT, foi desenvolvido um exemplo de uso práticos e documentação detalhada. O exemplo de uso abrangem cenários comuns de aplicação do WoTPy, demonstrando como utilizar suas funcionalidades e integrar dispositivos IoT. A documentação detalhada fornece informações sobre a instalação, configuração, e melhores práticas para aproveitar ao máximo o WoTPy. 

\section{Cronograma}

O trabalho foi conduzido em etapas sequenciais, com um cronograma definido para orientar o progresso e a organização das atividades, como demonstrado na figura \ref{fig:cronograma}.

Cada etapa foi planejada para ser concluída dentro de um período de tempo específico, permitindo o avanço contínuo do trabalho e o cumprimento dos prazos estabelecidos.

%
% https://www.overleaf.com/latex/examples/gantt-charts-with-the-pgfgantt-package/jmkwfxrnfxnw
% A fairly complicated example from section 2.9 of the package
% documentation. This reproduces an example from Wikipedia:
% http://en.wikipedia.org/wiki/Gantt_chart
%

\definecolor{barblue}{RGB}{153,204,254}
\definecolor{groupblue}{RGB}{51,102,254}
\definecolor{linkred}{RGB}{165,0,33}
\renewcommand\sfdefault{phv}
\renewcommand\mddefault{mc}
\renewcommand\bfdefault{bc}
\setganttlinklabel{s-s}{START-TO-START}
\setganttlinklabel{f-s}{FINISH-TO-START}
\setganttlinklabel{f-f}{FINISH-TO-FINISH}
\sffamily
\begin{figure}
    \caption{Cronogrma}
    \label{fig:cronograma}
\begin{adjustbox}{width=1.1\textwidth,center}
\begin{ganttchart}[
    canvas/.append style={fill=none, draw=black!5, line width=.75pt},
    hgrid style/.style={draw=black!5, line width=.75pt},
    vgrid={*1{draw=black!5, line width=.75pt}},
    today=4,
    today rule/.style={
      draw=black!64,
      dash pattern=on 3.5pt off 4.5pt,
      line width=1.5pt
    },
    today label font=\small\bfseries,
    title/.style={draw=none, fill=none},
    title label font=\bfseries\footnotesize,
    title label node/.append style={below=7pt},
    include title in canvas=false,
    bar label font=\mdseries\small\color{black!70},
    bar label node/.append style={left=2cm},
    bar/.append style={draw=none, fill=black!63},
    bar incomplete/.append style={fill=barblue},
    bar progress label font=\mdseries\footnotesize\color{black!70},
    group incomplete/.append style={fill=groupblue},
    group left shift=0,
    group right shift=0,
    group height=.5,
    group peaks tip position=0,
    group label node/.append style={left=.6cm},
    group progress label font=\bfseries\small,
    link/.style={-latex, line width=1.5pt, linkred},
    link label font=\scriptsize\bfseries,
    link label node/.append style={below left=-2pt and 0pt}
  ]{1}{12}
  \gantttitle[
    title label node/.append style={below left=7pt and -3pt}
  ]{MESES:\quad1}{1}
  \gantttitlelist{2,...,12}{1} \\
  
  \ganttgroup[progress=100]{WBS 1 Primeiro Semestre}{4}{7} \\
  \ganttbar[
    progress=100,
    name=WBS1A
  ]{\textbf{WBS 1.1} Analisar Especificações do W3C-WoT}{4}{4} \\
  \ganttbar[
    progress=100,
    name=WBS1B
  ]{\textbf{WBS 1.2} Analisar Dependências do WoTPy}{4}{4} \\
  \ganttbar[
    progress=100,
    name=WBS1C
  ]{\textbf{WBS 1.3} Resolver Problemas de Instalação}{5}{6} \\
  \ganttbar[
    progress=100,
    name=WBS1D
  ]{\textbf{WBS 1.4} Realizar Testes de Validação}{5}{6} \\
  \ganttbar[
    progress=100,
    name=WBS1E
  ]{\textbf{WBS 1.5} Desenvolver Exemplos de Uso}{6}{7} \\
  
  \ganttgroup[progress=0]{WBS 2 Segundo Semestre}{8}{12} \\
    \ganttbar[
    progress=0,
    name=WBS2A
  ]{\textbf{WBS 2.1} Capacitar em Grafos de Conhecimento}{8}{8} \\
  \ganttbar[
    progress=0,
    name=WBS2B
  ]{\textbf{WBS 2.2} Escolher Plataforma}{9}{9} \\
  \ganttbar[
    progress=0,
    name=WBS2C
  ]{\textbf{WBS 2.3} Armazenar Capacidades do Sensor e Observações}{9}{10} \\
  \ganttbar[
    progress=0,
    name=WBS2D
  ]{\textbf{WBS 2.4} Publicizar Grafos}{10}{12} \\
  \ganttbar[
    progress=0,
    name=WBS2E
  ]{\textbf{WBS 2.5} Definir e Implementar Casos de Teste}{12}{12} \\

  \ganttlink[link type=s-s]{WBS1A}{WBS1B}
  \ganttlink[link type=f-s]{WBS1B}{WBS1C}
  \ganttlink[link type=f-f]{WBS1C}{WBS1D}
  \ganttlink[link type=f-s]{WBS1D}{WBS1E}

  \ganttlink[link type=f-s]{WBS2A}{WBS2B}
  \ganttlink[link type=f-s]{WBS2B}{WBS2C}
  \ganttlink[link type=f-s]{WBS2C}{WBS2D}
  \ganttlink[link type=f-s]{WBS2D}{WBS2E}
  
  % \ganttlink[
    % link type=f-f,
    % link label node/.append style=left
  % {WBS1C}{WBS1D}
  
\end{ganttchart}
\end{adjustbox}
\end{figure}

